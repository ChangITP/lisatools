\documentclass[11pt]{article}
\usepackage{geometry}
\geometry{letterpaper,vmargin=1.0 in,hmargin=1.0 in}
\usepackage{graphicx}
\usepackage{amssymb}
\usepackage{epstopdf}
\DeclareGraphicsRule{.tif}{png}{.png}{`convert #1 `dirname #1`/`basename #1 .tif`.png}

\title{Challenge 3 Specification}
\author{MLDC taskforce}
\date{Dec 6, 2007}

\begin{document}
\maketitle

\subsection*{3.1: Galaxy with chirping binaries}

A two-year dataset ($2^{22}$ samples with $15$ s sampling time) with signals from $\sim 3 \times 10^7$ Galactic binaries.

Binaries are drawn from a randomized Nelemans population [\textbf{Neil: provide ref., description}], and orbital frequency may be increasing or decreasing depending on whether gravitational radiation or mass transfer is the dominant mechanism. The evolution of frequency is always linear (i.e., $\dot{f}$ is a relevant parameter, but $\ddot{f}$ is not). The set includes also 5--15 [\textbf{Shane: provide list}] verification binaries chosen from the original Nelemans population, with randomized extrinsic parameters.

Instrument noise is secondary-only, Gaussian, stationary, and equal on all spacecraft. 

\subsection*{3.2: Massive--Black-Hole binaries over Galactic confusion}

A two-year dataset ($2^{22}$ samples with $15$ s sampling time) with signals from 4--6 (with equal probabilities) spinning MBH binary inspirals.

The spinning-binary signals are modeled as 2PN circular adiabatic inspirals with uncoupled orbital frequency evolution and spin and orbital precession. No higher-PN harmonics are included. [\textbf{Neil and Stas: add more detail, link to full descriptive document}] The end of the inspiral is handled with an exponential taper, as in Challenge 2. Masses, SNRs, and times of coalescence are chosen as in Challenge 2; spin amplitudes are drawn uniformly between 0 and maximal, and spin angles are randomized over spheres. Sky position, polarization, extrinsic parameters are random over spheres.

The dataset includes also a Galactic confusion background generated from the same population as Challenge 3.1, by withholding all binaries with individual SNR [\textbf{Neil: TDI X? Sky-averaged? What noise curve?}] greater than XX [\textbf{Neil?}].

Instrument noise is secondary-only, Gaussian, stationary, and equal on all spacecraft. 

\subsection*{3.3.X: EMRIs}

Five two-year datasets ($2^{22}$ samples with $15$ s sampling time), each with a signal from an ``MLDC EMRI'', with parameters drawn as in Challenge 1B [\textbf{Excerpt MLDC 1B description here.}], but with SNRs uniformly distributed between 10 and 30. [\textbf{A previous discussion has SNR in 10--50, central masses all around $10^6$. Should we go with that?}] The number of eccentric-orbit harmonics does not vary along the evolution (as in Challenge 1B), but is fixed to 4 [\textbf{Stas: check}]. Sky position, polarization, extrinsic parameters are random over spheres.

Instrument noise is secondary-only, Gaussian, stationary, and equal on all spacecraft. 

\subsection*{3.4: Bursts}

One ``MLDC month''-long dataset ($2^{21}$ samples with 1 s sampling time) with Poisson-distributed cosmic-string--cusp bursts, defined as in the accompanying document [\textbf{To be included as a chapter here}]. The Poisson even rate is 5 per ``MLDC month''. SNRs will be uniformly distributed between 10 and 100. The logarithm of the maximum frequency (see accompanying document) will be uniformly distributed as $\log_10 f_\mathrm{max} \in [-4,1.0]$. Sky position and polarization are random over spheres.

Instrument noise is Gaussian and stationary; it includes secondary noise, where the levels of proof-mass and photo-detector noises are randomized independently on each optical bench by a uniform draw in $[-20,+20]\%$ w.r.t.\ their nominal value; it also includes laser noise, also randomized, with nominal amplitude reduced to $10\times$ the nominal amplitude of secondary noises at 1 mHz [\textbf{Michele: check}.] The datasets include the standard $X$, $Y$, $Z$ TDI observables \emph{and} the 12 inter-spacecraft and inter-spacecraft raw phase measurements. The datasets will be distributed only as fractional-frequency--fluctuation time series (i.e., as \emph{Synthetic LISA} datasets).

\subsection*{3.5: Stochastic Backgrounds}

One ``MLDC month''-long dataset ($2^{21}$ samples with 1 s sampling time [\textbf{Michele: check}]) with an isotropic stochastic background with $S_h(f) \propto f^{-3}$, realized by placing 192 ``pseudosources'' (independent pseudorandom processes for both $h_+$ and $h_\times$, with $f^{-3}$ spectra) at Healpix-distributed points across the sky [\textbf{Michele: ref.}]. The level of the background is raised to 3--6 times the secondary noise at the best frequency [\textbf{Michele: check}]

The instrument noise and the distributed observables are handled as in Challenge 3.4. The datasets will be released in versions produced by \emph{Synthetic LISA} and by the APC's \emph{LISACode}.

\end{document}  