\documentclass[12pt]{article}
\usepackage{times,latexsym}
\usepackage{amsmath}
\usepackage{stmaryrd}
\usepackage{graphicx}
\usepackage{epsfig}

\pagestyle{empty}
\setlength{\parindent}{0.5cm}
\setlength{\textwidth}{6.5in}
\setlength{\textheight}{8.6in}
\setlength{\oddsidemargin}{0in}
\setlength{\leftmargin}{0.1in}
\setlength{\parskip}{0cm}
\setlength{\evensidemargin}{0in}
\setlength{\topmargin}{0.1in}

\begin{document} 

\centerline{\Large{Waveform Description for Cosmic String Cusps}}

\bigskip

Cusps are regions on long strings or loops that achieve huge Lorentz
boosts $\gamma$.
These boosts, combined with the large mass per unit length of cosmic
strings, lead to the production of a gravitational radiation. The tip of
the cusp moves at the speed of light, and a small region of string
around the cusp has an enormous Lorentz boost. An analytical treatment
of the gravitational wave bursts produced by cusps has been provided
by Damour and Vilenkin~\cite{cusp1}. In the limit where the tip of the
cusp is moving directly toward the observer the resulting
metric perturbation is a linearly polarized gravitation wave with
\begin{equation}\label{cusp}
h(t) = A \vert t - t_\star \vert^{1/3} \, ,
\end{equation}
where $t_\star$ is the central time of arrival of the burst and the
amplitude $A$ is given by~\cite{cusp2}
\begin{equation}
A \sim \frac{G \mu L^{2/3}}{D_L} \, .
\end{equation}
Here $G$ is Newton's constant, $\mu$ is the mass per unit length of the
string, $D_L$ is the luminosity distance to the source, and $L$
is the size of the feature that produces the cusp (eg. the length of
a cosmic string loop). If the observer's line of sight does not coincide
with the direction that the cusp is moving, the waveform becomes much more
complicated, with a mixture of $+$ and $\times$
polarizations~\cite{cusp3}. If the viewing angle $\alpha$ departs
only slightly from zero, the waveform remains dominantly linearly
polarized, and the sharp spike in the waveform (\ref{cusp}) is rounded
off. In the Fourier domain this has the effect of introducing an
exponential suppression of power for frequencies above $f_{\rm max}$,
where $f_{\rm max}$ is related to the viewing angle $\alpha$ and
feature length $L$ by
\begin{equation}
f_{\rm max} = \frac{2}{\alpha^3 L} \, .
\end{equation}

Following the model used by the LSC, we define our cusp waveforms
in the Fourier domain according to
\begin{equation}
h(f)
\begin{cases} 
{\cal A} \left(1 + \left(\frac{f_{\rm low}}{f}\right)^2\right)^{-4}
\, f^{-4/3}  & f \leq f_{\rm max} \\
{\cal A} \left(1 + \left(\frac{f_{\rm low}}{f}\right)^2\right)^{-4}
\exp\left( 1 - \frac{f}{f_{\rm max}}\right) \, f^{-4/3}& f > f_{\rm max}
\end{cases}
\end{equation}
The amplitude ${\cal A}$ has dimensions ${\rm Hz}^{1/3}$, and $f_{\rm low}$
sets the low frequency cut-off of what is effectively a fourth order
Butterworth filter. This high pass filter prevents dynamic range issues
with the inverse Fourier transforms. For the MLDC data sets a reasonable
value to use is $f_{\rm low} = 1 \times 10^{-5}$ Hz.

The phase of the waveform is set using
\begin{eqnarray}
h_R(f) & = & -h(f)\cos(2 \pi f t_\star) \nonumber \\
h_I(f) & = & h(f) \sin(2 \pi f t_\star) \, .
\end{eqnarray}
The waveform is then inverse Fourier transformed to the time domain
to produce $h(t)$.
The final step needed to produce the Barycenter waveforms used by
the simulators is to apply a polarization rotation by angle $\psi$:
\begin{eqnarray}
h_+(t) & = & h(t) \cos(2 \psi) \nonumber \\
h_\times(t) & = & -h(f) \sin(2 \psi) \, .
\end{eqnarray}
The orientation of the various LISA arms imparts positional information
into the measured waveforms, which adds the ecliptic latitude and longitude
to the collection of parameters required to describe the signal from
a cosmic string cusp.

To summarize, a cosmic string cusp is described by the six parameters:
Amplitude ${\cal A}$, Central Time $t_\star$, Maximum Frequency $f_{\rm max}$,
Polarization $\psi$, Ecliptic Latitude and Ecliptic Longitude.

The amplitude should be chosen to give the desired matched filtered SNR.
The central time should be chosen uniformly in the observation period.
The maximum frequency should be chosen between $f_{\rm min}$ and
$f_{\rm Nyquist}$ (Once $f_{\rm max}$ exceeds the Nyquist frequency it
can not be determined from the data, so to limit aliasing it is a good
idea to cap its maximum value at the Nyquist frequency). The polarization
angle is uniform in the range $[0,\pi]$, and the sky location should be
uniform in longitude and the sine of the latitude.


\begin{thebibliography}{99}

\bibitem{cusp1} T. Damour and A. Vilenkin, Phys. Rev. Lett. {\bf 85}, 3761 (2000).
\bibitem{cusp2} X. Siemens, J. Creighton, I. Maor, S. Ray Majumder,
K. Cannon, and J. Read, Phys. Rev. D{\bf 73} 105001 (2006).
\bibitem{cusp3} X. Siemens and K. D. Olum, Phys. Rev. D{\bf 68}, 085017 (2003).

\end{thebibliography}



\end{document}
