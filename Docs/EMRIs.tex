%***************************************************************
% With Leor's compiler activate this:

%\documentstyle[floats,aps,epsf,amsfonts,amsmath,amssymb]{revtex}
%\tighten
%***************************************************************

%*******************************************************************
% With Curts's compiler activate this:

\documentclass[showpacs,preprintnumbers,amsmath,amssymb]{revtex4}
\usepackage{dcolumn}% Align columns on decimal point
\usepackage{bm}% bold math
\usepackage{graphicx}
\usepackage{dcolumn}% Align table columns on decimal point
\usepackage{bm}% bold math
\usepackage{graphicx}
%\usepackage{pdfsync}
%*******************************************************************

\def\etal{{\it et al.}}  \def\ie{{\it i.e.}}  \def\eg{{\it e.g.}}
\def\lap{\hbox{${_{\displaystyle<}\atop^{\displaystyle\sim}}$}}
\def\gap{\hbox{${_{\displaystyle>}\atop^{\displaystyle\sim}}$}}
\def\lesssim{\mathrel{\hbox{\rlap{\hbox{\lower4pt\hbox{$\sim$}}}\hbox{$<$}}}}
\def\gtrsim{\mathrel{\hbox{\rlap{\hbox{\lower4pt\hbox{$\sim$}}}\hbox{$>$}}}}
\def\alt{\mathrel{\hbox{\rlap{\hbox{\lower4pt\hbox{$\sim$}}}\hbox{$<$}}}}
\def\agt{\mathrel{\hbox{\rlap{\hbox{\lower4pt\hbox{$\sim$}}}\hbox{$>$}}}}
\newcommand {\be} {\begin{equation}} \newcommand {\ee}
{\end{equation}} \newcommand {\ban} {\begin{eqnarray}} \newcommand
{\ean} {\end{eqnarray}}

\begin{document}

\draft

\title{Analytic kludge waveforms for extreme-mass-ratio inspirals}

%*******************************************************************
% With Leor's compiler activate this:

%\author{Curt Cutler}
%\address
%{Jet Propulsion Lab}
%*******************************************************************

%*******************************************************************
% With Curts's compiler activate this:

%\author{Leor Barack}
%\email{leor@phys.utb.edu}
%\affiliation
%{Department of Physics and Astronomy and
%Center for Gravitational Wave Astronomy,
%University of Texas at Brownsville,
%Brownsville, Texas 78520}
\author{Curt Cutler}
% \email{cutler@aei.mpg.de}
%  \affiliation
%{Max-Planck-Institut fuer Gravitationsphysik,
%Albert-Einstein-Institut, Am Muehlenberg 1,
%D-14476 Golm bei Potsdam, Germany}
%*******************************************************************

\date{\today}
\maketitle

%\narrowtext

\section{Introduction}
This note summarizes the analytic kludge waveforms of Barack \& Cutler ~\cite{BC}, hereafter
referred to as BC.
The orbit is approximated as a Newtonian ellipse at any instant, but one whose
perihelion direction, orbital plane, semi-major axis, and eccenticity evolve 
%(both perihelion precession and Lense-Thirring precession of the orbital plane),
%inspiralling and varying in eccentricity 
according to post-Newtonian evolution equations.  At any instant, the emitted waveform
is taken to be the Peters-Matthews waveform corresponding to the instantaneous
Newtonian orbit.  While these waveforms are not particularly accurate in the
highly relativistic regime of interest for EMRI searches, they do exhibit the main qualitative
features of true waveforms, and are considerably simpler to generate.   It is expected that
any search strategy that works for "analytic kludge" waveforms could be converted over fairly
easily to true, GR waveforms, once these become available.
For more details we refer the reader to  BC (from which pieces were chopped up and
re-pasted to make this note).


Our index notation is the following. Indices for vectors and
tensors on parameter space are chosen from the beginning of the
Latin alphabet ($a,b,c,\ldots$).
Vectors and tensors on three-dimensional space have indices
chosen from the middle of the Latin alphabet ($i,j,k,\ldots$),
and run over $1,2,3$; their indices are raised and lowered with
the flat 3-metric, $\eta_{ij}$.
%We use Greek indices ($\alpha,\beta,\ldots$), running only over
%$I,II$, to label the two independent gravitational waveforms that
%LISA effectively generates. (No four-dimensional, spacetime indices
%occur in this paper.)
Actually,  we adopt a mixed notation for spatial vectors, sometimes
labelling them with spatial indices ($i,j,k,\ldots$), but sometimes
suppressing the indices and instead
using the standard $3-d$ vector notation: an over-arrow (as in $\vec A$)
to represent a vector, $\vec A \cdot \vec B$ to represent
a scalar (``dot'') product, and $\vec A \times \vec B$
to represent the vector (``cross'') product.
%$\equiv \epsilon^{ijk}A_j B_k$
An over-hat (as in $\hat n$)
will indicate that a vector is normalized, i.e., has unit length. We trust our
meaning will always be clear, despite this mixed notation.
Throughout this note we use units in which $G=c=1$.



%*********************************************************************
\section{Parameter space}
%*********************************************************************
The two-body system is described by 17 parameters. The spin of the
CO can be marginally relevant (see Appendix C of BC), but in this paper
we shall ignore it, leaving us with 14 parameters.
We shall denote a vector in the 14-d parameter space by $\lambda^a$
($a=0,\ldots,13$).
%where hereafter letters from the lower Latin alphabet ($a,b,\ldots$)
%are used as parameter-space indices.
We choose our parameters as follows:
%~~~~~~~~~~~~~~~~~~~~~~~~~~~~~~~~~~~~~~~~~~~~~~~~~~~~~~~~~~~~~~~~
\begin{eqnarray} \label{lambda}
\lambda^a &\equiv& (\lambda^0,\ldots,\lambda^{13}) =\nonumber\\
&&
%\left[t_0\nu_0,\,\ln\mu,\,\ln M,\,S/M^2,\,e_0,\,\tilde\gamma_0,\,\Phi_0,\,
\left[t_0\,\ln\mu,\,\ln M,\,S/M^2,\,e_0,\,\tilde\gamma_0,\,\Phi_0,\,
\mu_S\equiv\cos\theta_S,\,\phi_S,\,\cos\lambda,\,\alpha_0,
\mu_K\equiv\cos\theta_K,\,\phi_K,\,\ln(\mu/D)\right].
\end{eqnarray}
%~~~~~~~~~~~~~~~~~~~~~~~~~~~~~~~~~~~~~~~~~~~~~~~~~~~~~~~~~~~~~~~~
Here, $t_0$ is a time parameter that allows us to specify ``when''
the inspiral occurs---we shall generally choose $t_0$ to be the instant of
time when the (radial) orbital frequency sweeps through
some fiducial value $\nu_0$ (typically, we shall choose $\nu_0$ of
order $1\,$mHz),
$\mu$ and $M$ are the masses of the CO and MBH, respectively, and $S$ is the
magnitude of the MBH's spin angular momentum (so $0 \le S/M^2 \le 1$).
%$e_0$ is the orbital eccentricity at time $t_0$; $\tilde\gamma_0$ is the
%angle (in the plane of the orbit) from $\hat L \times \hat S$ to
%periastron, at $t_0$; $\Phi_0$ is the mean anomaly (i.e., the phase
%of the orbit, with respect to perihelion--true def, Curt??) at $t_0$,
The parameters $e_0$, $\tilde\gamma_0$, and $\Phi_0$ describe,
respectively, the eccentricity, the direction of the pericenter within
the orbital plane, and the mean anomaly---all at time $t_0$.
More specifically, we take $\tilde\gamma_0$ to be the angle (in the plane
of the orbit) from $\hat L \times \hat S$ to pericenter, and, as usual,
$\Phi_0$ to be the mean anomaly with respect to pericenter passage.
The parameter $\alpha_0\equiv\alpha(t=t_0)$ [where $\alpha(t)$
is defined in Eq.~(\ref{alpha})] describes the direction
of $\hat L$ around $\hat S$ at $t_0$.
The angles $(\theta_S,\phi_S)$ are the direction to the source, in
ecliptic-based coordinates; $(\theta_K,\,\phi_K)$ represent the direction
$\hat S$ of the MBH's spin (approximated as constant) in ecliptic-based
coordinates; and $\lambda$ is the angle between $\hat L$ and $\hat S$ (also
approximated as constant\footnote
{In reality, radiation reaction will impose a small time variation in
$\lambda$; however, this variation is known to be very small
(See Ref.\ \cite{scott1}) and we shall ignore
it here. When a model of the time-variation of $\lambda$ is eventually
at hand, it would be trivial to generalize our treatment to
incorporate it: one would just need an equation for $d\lambda/dt$,
and in the parameter list $\lambda$ would be replaced by
$\lambda_0$---the value of $\lambda$ at time $t_0$.}).
Finally, $D$ is the distance to the source.
%; for future convenience
%we take the dimensionless quantity $\ln(\mu/D)$ as our last parameter.

The various parameters and their meaning are summarized in Table
\ref{tableI}. Fig.\ \ref{fig1} illustrates the various angles involved
in our parameterization.

%------------------------------    FIGURE   -------------------------------
\begin{figure}[htb]
\input{epsf}
\centerline{\epsfysize 6cm \epsfbox{system.eps}}
\caption{\protect\footnotesize
The MBH-CO system: setup and notation.
$M$ and $\mu$ are the masses of the MBH and the CO, respectively.
The axes labeled $x{-}y{-}z$ represent a Cartesian system {\em based on
ecliptic coordinates} (the Earth's motion around
the Sun is in the x--y plane). The spin $\vec S$ of the MBH is parametrized
by its magnitude $S$ and the two angular coordinates $\theta_K,\phi_K$,
defined (in the standard manner) based on the system $x{-}y{-}z$.
$\vec L(t)$ represents the (time-varying) orbital angular momentum;
its direction is parametrized by the (constant) angle $\lambda$ between
$\vec L$ and $\vec S$, and by an azimuthal angle $\alpha(t)$ (not
shown in the figure).
The angle $\tilde\gamma(t)$ is the (intrinsic) direction of pericenter,
as measured with respect to $\vec L\times\vec S$. Finally,
$\Phi(t)$ denotes the mean anomaly of the orbit, i.e., the average
orbital phase with respect to the direction of pericenter.}
\label{fig1}
\end{figure}
%--------------------------------------------------------------------------

%----------------------------- TABLE I ----------------------------------
\begin{table}[thb]
\centerline{$\begin{array}{c|c|l}\hline\hline
%\lambda^0 & t_0\nu_0      & \text{$t_0$ is instant at which orbital frequency
\lambda^0 & t_0      & \text{$t_0$ is time where orbital frequency
    sweeps through fiducial value (e.g., 1mHz)}   \\
\lambda^1 & \ln\mu        & \text {($\ln$ of) CO's mass}\\
\lambda^2 & \ln M         & \text {($\ln$ of) MBH's mass}\\
\lambda^3 & S/M^2         & \text{magnitude of (specific) spin
    angular momentum of MBH} \\
\lambda^4 & e_0           & \text{$e(t_0)$, where $e(t)$ is the
    orbital eccentricity} \\
\lambda^5 & \tilde\gamma_0 &  \text{$\tilde\gamma(t_0)$,
    where $\tilde\gamma(t)$ is the angle (in orbital plane)
    between $\hat L\times\hat S$ and pericenter}      \\
\lambda^6 & \Phi_0        & \text{$\Phi(t_0)$, where $\Phi(t)$ is
    the mean anomaly}\\
\lambda^7 & \mu_S\equiv\cos\theta_S  & \text{(cosine of) the source direction's
    polar angle }  \\
\lambda^8 & \phi_S        & \text{azimuthal direction to source}  \\
\lambda^9 & \cos\lambda   & \hat L\cdot\hat S(={\rm const}) \\
\lambda^{10} & \alpha_0     & \text{$\alpha(t_0)$, where $\alpha(t)$
    is the azimuthal direction of $\hat L$ (in the orbital plane)}   \\
\lambda^{11} & \mu_K\equiv\cos\theta_K & \text{(cosine of) the polar angle
   of MBH's spin}  \\
\lambda^{12} & \phi_K       &  \text{azimuthal direction of MBH's
    spin}  \\
\lambda^{13} & \ln(\mu/D)          & \text{($\ln$ of) CO's mass divided by distance to source}\\
\hline\hline
\end{array}$}
\caption{\protect\footnotesize
Summary of physical parameters and their meaning.
The angles ($\theta_S$,$\phi_S$) and ($\theta_K$,$\phi_K$)
are associated with a spherical coordinate system attached to the ecliptic.
$\hat L$ and $\hat S$ are unit vectors in the directions
of the orbital angular momentum and the MBH's spin, respectively.
For further details see figure \ref{fig1} and the description
in the text.}\label{tableI}
\end{table}
%-------------------------------------------------------------------------

Note for simplicity we are treating the background spacetime as
Minkowski space, not Robertson-Walker. To correct this, for a source
at redshift $z$, requires only
the simple translation: $M\rightarrow M(1+z)$, $\mu\rightarrow \mu(1+z)$,
$S\rightarrow S(1+z)^2$, $D\rightarrow D_L$, where $D_L$ is the
``luminosity distance''~\cite{markovic}.

The parameters can be divided into ``intrinsic'' and ``extrinsic''
parameters, following  Buonanno, Chen, and
Vallisneri~\cite{BCV} (hereafter, BCV).
Extrinsic parameters refer to the observer's position or orientation, or to
the zero-of-time on the observer's watch.
There are seven extrinsic parameters: the four parameters
$t_0$,  $\mu_S$, $\phi_S$, and $D$ correspond to the
freedom to translate the same binary in space and time, and
the three parameters $\mu_K$, $\phi_K$, and $\alpha_0$ are
basically Euler angles that specify the orientation of the
orbit with respect to the observer (at $t_0$).
The intrinsic parameters are the ones that control the detailed
dynamical evolution of the system, without reference to the
observer's location or orientation.
In our parametrization, the seven intrinsic parameters are
$\ln\mu$, $\ln M$, $S/M^2$, $\cos\lambda$, $e_0$, $\tilde\gamma_0$,
and $\Phi_0$.
BCV observed (in the context of circular-orbit binaries with spin) that
extrinsic parameters are generally much ``cheaper'' to search over
than intrinsic parameters, which can be important for constructing
efficient search strategies.


%*********************************************************************
\section{Principal axes}
%*********************************************************************


Let ${\hat n}$ be the unit vector pointing from the detector
to the source, and let ${ \hat L}(t)$ be the unit vector along
the CO's orbital angular momentum.
We find it convenient to work in a (time-varying) wave frame defined with
respect to ${ \hat n}$ and ${ \hat L}(t)$. We define unit vectors
${ \hat p}$ and ${ \hat q}$ by
\begin{eqnarray} \label{pq}
{ \hat p} &\equiv& ({ \hat n}\times { \hat L})/
                    |{ \hat n}\times { \hat L}|, \nonumber\\
{ \hat q} &\equiv& { \hat p} \times { \hat n},
\end{eqnarray}
%~~~~~~~~~~~~~~~~~~~~~~~~~~~~~~~~~~~~~~~~~~~~~~~~~~~~~~~~~~~~~~~~
based on which we then
define the two polarization basis tensors
%~~~~~~~~~~~~~~~~~~~~~~~~~~~~~~~~~~~~~~~~~~~~~~~~~~~~~~~~~~~~~~~~
\begin{eqnarray} \label{H}
H_{ij}^{+}(t)      & \equiv & \hat p_i \hat p_j - \hat q_i \hat q_j, \nonumber\\
H_{ij}^{\times}(t) & \equiv & \hat p_i \hat q_j + \hat q_i \hat p_j.
\end{eqnarray}
%~~~~~~~~~~~~~~~~~~~~~~~~~~~~~~~~~~~~~~~~~~~~~~~~~~~~~~~~~~~~~~~~
The general GW strain field at the detector can then be written as
\begin{equation} \label{hab}
h_{ij}(t)=
A^{+}(t)H_{ij}^{+}(t) + A^{\times}(t)H_{ij}^{\times}(t),
\end{equation}
%~~~~~~~~~~~~~~~~~~~~~~~~~~~~~~~~~~~~~~~~~~~~~~~~~~~~~~~~~~~~~~~~
where $A^{+}(t)$ and $A^{\times}(t)$ are the amplitudes of the two
polarizations.  In  the next section we derive expressions for $A^{+}(t)$ and $A^{\times}(t)$.

\section{Peters-Mathews waveforms}

In the quadrupole approximation, the metric perturbation
far from the source is given (in the ``transverse/traceless''
gauge) by \cite{MTW}
\be\label{quad}
h_{ij} = (2/D)\bigl(P_{ik}P_{jl} -
\frac{1}{2}P_{ij}P_{kl}\bigr) \ddot I^{kl}
\ee
where %``TT'' stands for the ``transverse, traceless'' gauge,
$D$ is the distance to the source, the projection operator $P_{ij}$ is
given by
$P_{ij} \equiv \eta_{ij} - {\hat n}_i{\hat n}_j$, and
$\ddot I^{ij}$ is the second time derivative of the inertia tensor.
In this paper we work in the limit of small mass ratio, $\mu/M\ll 1$,
where $\mu$ and $M$ are the masses of the CO and MBH, respectively.
In this limit, the inertia tensor is just
%\be\label{inertia_tensor}
$I^{ij}(t) = \mu\, r^i(t) r^j(t)$,
%\ee
where $\vec r$ is the position vector of the CO with respect to the MBH.

Consider now a CO-MBH system described as a Newtonian binary, with semi-major
axis $a$, eccentricity $e$, and orbital frequency $\nu = (2\pi M)^{-1}
(M/a)^{3/2}$.
Let ${\hat e_1}$ and ${\hat e_2}$ be orthonormal vectors pointing
along the major and minor axes of the orbital ellipse, respectively.
Since the orbit is planar, $I^{ij}$ has only 3 independent components:
$I^{11}$, $I^{21}$, and $I^{22}$,
and as the motion is periodic, we can express $I^{ij}$ as a sum of harmonics
of the orbital frequency $\nu$:
$I^{ij}=\sum_n I^{ij}_n$.

We next denote
\ban
a_n &\equiv& {1\over 2}(\ddot I^{11}_n - \ddot I^{22}_n), \nonumber\\
b_n &\equiv& \ddot I^{12}_n, \nonumber\\
c_n &\equiv& {1\over 2}(\ddot I^{11}_n + \ddot I^{22}_n). \label{anbncn}
\ean
%where $(\ddot I_{ij})_n$ is the piece of $\ddot I_{ij}$ with
%frequency $n\nu$. Then
Peters and Matthews showed \cite{pm} that
%~~~~~~~~~~~~~~~~~~~~~~~~~~~~~~~~~~~~~~~~~~~~~~~~~~~~~~~~~~~~~~~~
\begin{eqnarray} \label{abc}
a_n &=& - n{\cal A}\bigl[J_{n-2}(ne)-2eJ_{n-1}(ne)+(2/n)J_n(ne)
+2eJ_{n+1}(ne)-J_{n+2}(ne)\bigr]\cos[n\Phi(t)],
\nonumber\\
b_n &=& - n{\cal A}(1-e^2)^{1/2}\bigl[J_{n-2}(ne)-2J_{n}(ne)
+J_{n+2}(ne)\bigr]\sin[n \Phi(t)],
\nonumber\\
c_n &=& 2{\cal A}J_n(ne)\cos[n\Phi(t)],
\end{eqnarray}
where
%~~~~~~~~~~~~~~~~~~~~~~~~~~~~~~~~~~~~~~~~~~~~~~~~~~~~~~~~~~~~~~~~
\begin{equation} \label{calA}
{\cal A}\equiv (2\pi \nu M)^{2/3}\frac{\mu}{D} ,
\end{equation}
%~~~~~~~~~~~~~~~~~~~~~~~~~~~~~~~~~~~~~~~~~~~~~~~~~~~~~~~~~~~~~~~~
$J_n$ are Bessel functions of the first kind,
and $\Phi(t)$ is the mean anomaly (measured from pericenter).
For a strictly Newtonian binary we have
%~~~~~~~~~~~~~~~~~~~~~~~~~~~~~~~~~~~~~~~~~~~~~~~~~~~~~~~~~~~~~~~~
\begin{equation} \label{Phi}
\Phi(t) = 2\pi\nu (t-t_0) +\Phi_0,
\end{equation}
%~~~~~~~~~~~~~~~~~~~~~~~~~~~~~~~~~~~~~~~~~~~~~~~~~~~~~~~~~~~~~~~~
where $\Phi_0$ is the mean anomaly at $t_0$.
Decomposing Eq.\ (\ref{hab}) into $n$-harmonic contributions and
using Eq.~(\ref{quad}), one then easily obtains explicit expressions
for the $n$-harmonic components of the two polarization
coefficients, 
\begin{eqnarray}\label{harmonics}
A^{+}\equiv \sum_n A^{+}_n \\
A^{\times}\equiv \sum_n A^{\times}_n \, . \\
\end{eqnarray}
where the $A^{+,\times}_n$ are
%~~~~~~~~~~~~~~~~~~~~~~~~~~~~~~~~~~~~~~~~~~~~~~~~~~~~~~~~~~~~~~~~
\begin{eqnarray} \label{A}
A^{+}_n &=&-[1+({ \hat L}\cdot{ \hat n})^2]\left[
a_n\cos(2\gamma)-b_n\sin(2\gamma)\right]
+[1-({ \hat L}\cdot{ \hat n})^2]c_n, \nonumber\\
A^{\times}_n&=& 2({ \hat L}\cdot{ \hat n})\left[
b_n \cos(2\gamma)+a_n \sin(2\gamma)\right],
\end{eqnarray}
%~~~~~~~~~~~~~~~~~~~~~~~~~~~~~~~~~~~~~~~~~~~~~~~~~~~~~~~~~~~~~~~~
where $\gamma$ is an azimuthal angle measuring the direction of
pericenter with respect to
$\hat x \equiv [-\hat n + \hat L (\hat L\cdot \hat n)]
/[1-(\hat L\cdot \hat n)^2]^{1/2}$.  In practice, we truncate the sums in Eq.~(\ref{harmonics})
at $n=20$.

The angular momenum direction vector $\hat L$ is not constant, since $\hat L$ precesses about the MBH's spin
direction $\hat S$.  Let $\theta_L(t),\phi_L(t)$ be the angles specifying the instantaneous
direction of $\hat L$.  These can be expressed in terms of the other angles in the problem
as follows.
Recall that  $\theta_K,\phi_K$  give the direction of $\vec S$ in the ecliptic-based
system (`K' standing for `Kerr'). 
Let $\lambda$ be the angle {\it between} $\hat L$ and $\hat S$, and let
$\alpha(t)$ be an azimuthal angle (in the orbital plane) that measures the
precession of $\hat L$ {\it around} $\hat S$:
Specifically, let
\be \label{alpha}
\hat L = \hat S \, \cos\lambda +
\frac{\hat z - \hat S \cos\theta_K}{\sin\theta_K} \sin\lambda \cos\alpha
+ \frac{\hat S \times \hat z}{\sin\theta_K}  \, \sin\lambda \sin\alpha,
\ee
\noindent
where $\hat z$ is a unit vector normal to the ecliptic. Then the angles
$\theta_L(t),\phi_L(t)$ are given in terms of $\theta_K$, $\phi_K$,
$\lambda$, $\alpha(t)$ by
%~~~~~~~~~~~~~~~~~~~~~~~~~~~~~~~~~~~~~~~~~~~~~~~~~~~~~~~~~~~~~~~~
\begin{eqnarray}\label{relations3}
\cos\theta_L(t) &=& \cos\theta_K \cos\lambda
    +\sin\theta_K\sin\lambda\cos\alpha(t), \nonumber\\
\sin\theta_L(t)\cos\phi_L(t) &=&
\sin\theta_K\cos\phi_K\cos\lambda
-\cos\phi_K\cos\theta_K\sin\lambda\cos\alpha(t)
+\sin\phi_K\sin\lambda\sin\alpha(t),  \nonumber\\
\sin\theta_L(t)\sin\phi_L(t) &=&
\sin\theta_K\sin\phi_K\cos\lambda
-\sin\phi_K\cos\theta_K\sin\lambda\cos\alpha(t)
-\cos\phi_K\sin\lambda\sin\alpha(t) .
\end{eqnarray}
%~~~~~~~~~~~~~~~~~~~~~~~~~~~~~~~~~~~~~~~~~~~~~~~~~~~~~~~~~~~~~~~~
The evolution equation for $\alpha(t)$ is given in Sec.\  \ref{EvEq} below.


%*********************************************************************
\subsection{The pericenter angle $\tilde \gamma$ }
%*********************************************************************

As mentioned above, the angle $\gamma$ that appears in Eqs.\ (\ref{A})
measures the
direction of pericenter with respect to
$\hat x \equiv [-\hat n + \hat L (\hat L\cdot \hat n)]/
[1-(\hat L\cdot \hat n)^2]^{1/2}$. With this definition,
$\gamma$ is neither purely extrinsic nor purely intrinsic.
(In the terminology of BCV, ``intrinsic'' parameters describe the
system without reference to the location or orientation of the observer.)
We will find it convenient to introduce a
somewhat different convention for the
zero-point of this angle: We shall define $\tilde\gamma$ to be
the direction of pericenter
with respect to $\hat L \times \hat S$. Then $\tilde\gamma$
is a purely intrinsic quantity.

Clearly, $\gamma$  and $\tilde \gamma$ are related by
\be\label{beta}
\gamma = \tilde\gamma + \beta,
%\beta = \gamma - \tilde\gamma \, ,
\ee
where $\beta$ is the angle from $\hat x \propto [\hat L(\hat L \cdot \hat n)
- \hat n] $ to $(\hat L \times \hat S)$.
It is straightforward to show that $\beta$ is given by
%(cf.\ App.\ A)
\begin{eqnarray}\label{sinbeta}
\sin\beta &=& \frac{\cos\lambda\, \hat L\cdot\hat n -\hat S\cdot \hat n }
{\sin\lambda\bigl[1 - (\hat L\cdot\hat n)^2\bigr]^{1/2}}, \nonumber \\
\cos\beta &=& \frac{\hat n \cdot (\hat S \times \hat L)}
{\sin\lambda\bigl[1 - (\hat L\cdot\hat n)^2\bigr]^{1/2} }.
\end{eqnarray}
To evaluate $\beta(t)$ in practice, it is useful to know the following
relations:
\be\label{SdotN}
{ \hat S}\cdot{ \hat n} = \cos\theta_S \cos\theta_K
+ \sin\theta_S \sin\theta_K \cos(\phi_S-\phi_K),
\ee
\be\label{ScrossLdotN}
\hat n \cdot (\hat S \times \hat L) =
\sin\theta_S \sin(\phi_K-\phi_S)\sin\lambda \cos\alpha
+ \frac{\hat S\cdot\hat n \cos\theta_K -\cos\theta_S}{\sin\theta_K}
\, \sin\lambda \sin\alpha,
\ee
and
\be\label{LdotN}
{ \hat L}\cdot{ \hat n} = { \hat S}\cdot{ \hat n}\cos\lambda
+ \frac{\cos\theta_S - \hat S\cdot\hat n \cos\theta_K}{\sin\theta_K}
\, \sin\lambda \cos\alpha + \frac{(\hat S \times \bar z)\cdot \hat n}
{\sin\theta_K}  \, \sin\lambda \sin\alpha,
\ee
or, equivalently,
\be\label{2LdotN}
{ \hat L}\cdot{ \hat n} = \cos\theta_S \cos\theta_L +
\sin\theta_S \sin\theta_L \cos(\phi_S-\phi_L).
\ee
Note that the time-variation of ${ \hat S}\cdot{ \hat n}$ is very small
in the extreme mass-ratio case considered here.
In our kludged model we approximate $\hat S$---and hence ${\hat S}\cdot{\hat n}$---as strictly constant.


\section{Orbital evolution equations}
\label{EvEq}

We evolve $\Phi(t)$, $\nu(t)$, $\tilde\gamma(t)$, $e(t)$,
and $\alpha(t)$ using the following PN formulae:
%[[give refs]]
%[[split up gamma in 2 parts]]
%~~~~~~~~~~~~~~~~~~~~~~~~~~~~~~~~~~~~~~~~~~~~~~~~~~~~~~~~~~~~~~~~
%\begin{mathletters} \label{PN}
\begin{eqnarray}
\frac{d\Phi}{dt} &=& 2\pi\nu, \label{Phidot} \\
%
%\frac{d\nu}{dt} &=& \frac{96}{10\pi}(\mu/M^3)(2\pi M\nu)^{11/3}\nonumber \\
%&&\times \left[1+(73/24)e^2 + (37/96)e^4\right](1-e^2)^{-7/2}[1 + PN]
%\label{nudot} \\
%%
\frac{d\nu}{dt} &=&
\frac{96}{10\pi}(\mu/M^3)(2\pi M\nu)^{11/3}(1-e^2)^{-9/2}
\bigl\{
\left[1+(73/24)e^2+(37/96)e^4\right](1-e^2) \nonumber \\
&&+ (2\pi M\nu)^{2/3}\left[(1273/336)-(2561/224)e^2-(3885/128)e^4
-(13147/5376)e^6 \right] \nonumber \\
&&- (2\pi M\nu)(S/M^2)\cos\lambda (1-e^2)^{-1/2}\bigl[(73/12)
+ (1211/24)e^2 \nonumber \\
&&+(3143/96)e^4 +(65/64)e^6 \bigr]
\bigr\}, \label{nudot} \\
%
\frac{d\tilde\gamma}{dt} &=& 6\pi\nu(2\pi\nu M)^{2/3} (1-e^2)^{-1}
\left[1+\frac{1}{4}(2\pi\nu M)^{2/3} (1-e^2)^{-1}(26-15e^2)\right] \nonumber \\
&&-12\pi\nu\cos\lambda (S/M^2) (2\pi M\nu)(1-e^2)^{-3/2},
\label{Gamdot} \\
%
\frac{de}{dt}  &=& -\frac{e}{15}(\mu/M^2) (1-e^2)^{-7/2} (2\pi M\nu)^{8/3}
\bigl[(304+121e^2)(1-e^2)\bigl(1 + 12 (2\pi M\nu)^{2/3}\bigr) \, \nonumber \\
&&- \frac{1}{56}(2\pi M\nu)^{2/3}\bigl( (8)(16705) + (12)(9082)e^2 - 25211e^4 \bigr)\bigr]\,
\nonumber \\
&&+ e (\mu/M^2)(S/M^2)\cos\lambda\,(2\pi M\nu)^{11/3}(1-e^2)^{-4}
\, \bigl[(1364/5) + (5032/15)e^2 + (263/10)e^4\bigr] ,
\label{edot} \\
%
\frac{d\alpha}{dt} &=& 4\pi\nu (S/M^2) (2\pi M\nu)(1-e^2)^{-3/2}.
\label{alphadot}
\end{eqnarray}
%\end{mathletters}
%~~~~~~~~~~~~~~~~~~~~~~~~~~~~~~~~~~~~~~~~~~~~~~~~~~~~~~~~~~~~~~~~
Equations (\ref{nudot}), (\ref{Gamdot}), and (\ref{edot}) are from
Junker and Sch\"afer~\cite{JunkerSchaefer}, except (i) the second line
of Eq.\ (\ref{Gamdot}) is from Brumberg~\cite{Brumberg}
(cf.\ our Appendix A),
%[in fact we use  now show in Appendix that it is equivalent
%to Brumberg]],
and the last term in Eq.\ (\ref{nudot})---the term $\propto S/M^2$---is from
Ryan~\cite{ryan96}.  Eq.~(\ref{alphadot}) is from Barker and
O'Connell~\cite{Barker}.
The dissipative terms $d\nu/dt$ and
$de/dt$ are given accurately through 3.5PN order (i.e., one order higher
than 2.5PN order, where radiation reaction first becomes manifest).
The non-dissipative equations, for $d\tilde\gamma/dt$ and $d\alpha/dt$,
are accurate through 2PN order.\footnote{In fact, the equations for
$d\tilde\gamma/dt$ and $d\alpha/dt$ are missing terms proportional
to $(S/M^2)^2$, which, according to usual ``order counting'' are
classified as 2PN. However, this usual counting is misleading when
the central object is a spinning BH: Because BHs are ultracompact, their spins
are smaller than suggested by the usual counting, and the missing terms
$\propto (S/M^2)^2$ have, in fact, the same magnitude as 3PN terms.
Similarly, the terms  $\propto (S/M^2)$ in Eqs.~(\ref{Gamdot}) and
(\ref{alphadot}) can be viewed as effectively 1.5PN terms.}

In solving the above time-evolution equations, the initial values (at time
$t_0$) of $\Phi$, $\nu$, $\tilde\gamma$, $e$, and $\alpha$ are just the parameters
$\Phi_0$, $\nu_0$, $\tilde\gamma_0$, $e_0$, and $\alpha_0$.

We use our PN Eqs.~(\ref{edot}) and (\ref{nudot})
to evolve $e(t)$ and $\nu(t)$ forward in time, up
to the point when the CO plunges over the top of the
effective potential barrier. For a point particle in Schwarzschild, the
plunge occurs at
$a_{\rm min} = M (6 + 2e)(1-e^2)^{-1}$~\cite{Cutler-Kennefick-Poisson},
so we set
\be\label{numax}
\nu_{\rm max} = (2\pi M)^{-1}[(1-e^2)/(6 + 2e)]^{3/2} \, ,
\ee
so we cut off the integration when $\nu$ reaches this $\nu_{\rm max}$.


\subsection{Doppler phase modulation}

Doppler phase modulation
due to LISA's orbital motion becomes important
for integration times longer than a few weeks.
We incorporate this effect by shifting the phase
$\Phi(t)$, according to
%~~~~~~~~~~~~~~~~~~~~~~~~~~~~~~~~~~~~~~~~~~~~~~~~~~~~~~~~~~~~~~~~
\begin{equation}\label{phid1}
\Phi(t)\to \Phi(t)+\Phi^D(t),
\end{equation}
%~~~~~~~~~~~~~~~~~~~~~~~~~~~~~~~~~~~~~~~~~~~~~~~~~~~~~~~~~~~~~~~~
where
%~~~~~~~~~~~~~~~~~~~~~~~~~~~~~~~~~~~~~~~~~~~~~~~~~~~~~~~~~~~~~~~~
\begin{equation}\label{phid2}
\Phi^D(t)=2\pi \nu(t) R \sin\theta_S \cos[2\pi(t/T) + \bar\phi_0-\phi_S].
\end{equation}
%~~~~~~~~~~~~~~~~~~~~~~~~~~~~~~~~~~~~~~~~~~~~~~~~~~~~~~~~~~~~~~~~
Here $R \equiv 1\, {\rm AU} = 499.00478$ sec, and $\bar\phi_0$ specificies
the location of the LISA detector (around the Sun) at time $t=0$.

\section{The polarization angle}
Eq.~(\ref{hab}) express the waveform in terms of polarization tensors $H^{\pm}_{ij}(t)$ that
are time-varying.  However to generate the LISA responses with Synthetic LISA or
LISA Simulator, it is useful to re-express Eq.~(\ref{hab}) in terms of fixed polarization tensors.
How do we do this? First note that at any instant, within the Synthetic LISA conventions, the
polarization angle is given by 
\begin{equation}\label{psiSL}
\psi_{SL} = -arctan\bigl(\frac{\cos \theta_S \sin \theta_L \cos(\phi_S - \phi_L) - \cos\theta_L \sin\theta_S}{\sin\theta_L \sin(\phi_S-\phi_L)}\bigr) \,.
\end{equation}

I.e., for this polarization angle, $h_+ = A^+$  and  $h_{\times} = 
A^{\times}$.
To use Synthetic LISA, though, we want to choose some fixed $\psi$-- call it $\psi_0$. 
(Again, this just  corresponds to fixing the basis of polarization tensors.  One could just 
set $\psi_0$ equal to $0$, but we'll allow it to be arbitrary here.) With respect to this new
basis, $h_{+,\times}(t)$ are given by
\begin{eqnarray}\label{final}
h_+(t) = A^+(t) cos(2\psi_0 - 2\psi_{SL}(t) ) + A^{\times}(t) sin(2\psi_0 - 2\psi_{SL}(t) ) \nonumber \\
h_{\times}(t) = A^{\times}(t) cos(2\psi_0 - 2\psi_{SL}(t) ) - A^+(t) sin(2\psi_0 - 2\psi_{SL}(t) )
\end{eqnarray}

[[Note: I'm actually not sure I understand Michele's sign convention wrt polarization angle, so
there could be some sign errors in this last bit.  I'll sort it out with him. CC]]

\section{Putting the pieces together}

The algorithm for constructing our approximate waveform is then:
Fix some fiducial frequency $\nu_0$ and choose waveform parameters
$(t_0,\,\ln\mu,\,\ln M,\,S/M^2,\,e_0,\,\tilde\gamma_0,\,\Phi_0,\,\cos\theta_S,\,
\phi_S,\,\cos\lambda,\,\alpha_0,\cos\theta_K,\,\phi_K,D)$.
Solve the ODEs (\ref{Phidot})--(\ref{alphadot})
for $\Phi(t)$, $\nu(t)$, $\tilde\gamma(t)$, $e(t)$, $\alpha(t)$.
Use $e(t)$ and $\nu(t)$ to
calculate $a_n(t), b_n(t), c_n(t)$ in Eqs.~(\ref{abc}),
remembering to include the Doppler modulation via
$n\Phi(t)\to n[\Phi(t)+\Phi^D(t)]$, a la Eqs.~(\ref{phid1}) and (\ref{phid2}).
Calculate $\theta_L(t),\phi_L(t)$ using Eqs.~(\ref{relations3}), and then
calculate $\gamma(t)$ from $\tilde\gamma(t)$ using Eqs.~(\ref{beta})--(\ref{2LdotN}).
Calculate the amplitude coefficients $A_n^{+,\times}$  using Eqs.~(\ref{A}) and (\ref{2LdotN}).
Calculate $\psi_{SL}$ using Eq.~(\ref{psiSL}).
Then finally calculate $h_{\pm}(t)$ (for the $\psi_0$ of your choice) using
Eqs.~(\ref{final}).

Note that, in our kludge treatment, pericenter precession and Lense-Thirring
precession have no effect on the $a_n, b_n, c_n$.  The effect of these motions is simply to
rotate the binary system, which modifies the amplitude harmonics (via Eqs.~\ref{A} ) and the 
also polarization angle $\psi_{SL}(t)$. The latter enters the time-dependence of
$h_{+,\times}(t)$ via Eqs.~(\ref{final}). 

\section{Choice of parameters for the first data challenge}

\section{Implementation}

First of all we should mention that some parameters in the code 
were fixed to the following values:
(i) the initial LISA's phase $\bar{\phi}_0$ was taken to be zero 
and (ii) $1yr$ is sidedreal year $= 31556925.2 (sec)$.

The waveform described here was implemented as C++ class with 
the following methods:
\begin{itemize}
\item {\tt AKWaveform(float spin, float mu, float MBHmass, float tfin, float timestep)}. This is constructor with parameters: {\tt spin}  is dimensional (reduced) spin of MBH (between 0 and 1), {\tt mu} is the mass of CO in $M_{\odot}$, {\tt MBHmass} is the mass of MBH
in $M_{\odot}$, {\tt tfin} is the duration time in sec (inital time is assumed to be zero), and {\tt timestep} defines sampling interval 
(in sec.). 
\item {\tt SetSourceLocation(float thS, float phS,  float thK, float phK, float D).} In this function user defines location of the source through
the following parameters: {\tt thS} this is $\theta_S$, {\tt phS}
this is $\phi_S$, {\tt thK} corresponds to $\theta_K$ and 
{\tt phK} to $\phi_K$, the distance to the source is {\tt D} and
it is given in pc.

\item {\tt EstimateInitialParams(float elso, float nulso, float ein, 
float nuin)}
This function estimates inital frequency {\tt nuin} and initial
eccentricity {\tt ein} for values given at plunge {\tt elso}
by integrating simplified evolution equations backwards.

\item {\tt EvolveOrbit(float nu0, float eccen, float gamma0, \
		    float Phi0, float al0, float lam)}. This method is used 
to evolve the orbit with initial conditions specified by 
{\tt nu0, eccen, gamma0, Phi0, al0} and for a given {\tt lam} 
$= \lambda$ 

\item {\tt GetOrbitalEvolution(Matrix<float>\& time, Matrix<float>\& Phit, Matrix<float>\& nut,  Matrix<float>\& gammat, Matrix<float>\& et, Matrix<float>\& alt)}. Using this function user can have a look at the result of integration of eqns. (\ref{Phidot} - \ref{alphadot})

\item {\tt GetWaveform(float ps0, Matrix<float>\& time, Matrix<float>\& hplus,  Matrix<float>\& hcross)}. Finally this function will 
fill up 1-d matrices (vectors) with waveforms with initial 
polarization angle specified by {\tt ps0}$= \psi_0$.
\end{itemize}
Note that orbital evolution is decoupled from the source location,
so one can compute waveform for many direction on the sky, 
for the same intrinsic parameters (multiple call to {\tt SetSourceLocation} and {\tt GetWaveform}, with only one run of
{\tt EvolveOrbit}).

\begin{thebibliography}{99}

\bibitem{BC} L.\ Barack and C. Cutler, Phys.\ Rev.\ {\bf 69}, 082005(2004).

\bibitem{ryan_multipoles}
    F.\ D.\ Ryan, Phys.\ Rev.\ D {\bf 56}, 1845 (1997).
%\bibitem{ryan_multipoles}
%    F.\ D.\ Ryan, Phys.\ Rev.\ D {\bf 56}, 1845 (1997);
%    {\bf 56}, 7732 (1997).

%\bibitem{Poisson_03}
%E.\ Poisson, Living Reviews in Relativity, submitted (gr-qc/0306052).

\bibitem{pm}
P.\ C.\ Peters and J.\ Mathews, Phys.\ Rev.\ {\bf 131}, 435 (1963);
P.\ C.\ Peters, Phys.\ Rev.\ {\bf 136}, B1224 (1964).

%\bibitem{Cutler-Kennefick-Poisson}
%C.\ Cutler, D.\ Kennefick and E.\ Poisson, Phys. Rev. D
%{\bf 50}, 3816 (1994).
%note Glampedakis-Kennefick attributes term ``zoom-whirl'' to me and Eric and
%maybe Kip, but I don't think we used it in our paper (though
% I haven't checked carefully)

%\bibitem{Glampedakis-Kennefick}
%K.\ Glampedakis and D.\ Kennefick, Phys.\ Rev.\ D {\bf 66}, 044002 (2002).

%\bibitem{scott1}
  %      S.\ A.\ Hughes, Phys. Rev. D {\bf 61}, 084004 (2000).

\bibitem{cutler98}
    C. Cutler, Phys. Rev. D. {\bf 57}, 7089 (1998).

%\bibitem{Poisson96}
 %   E.\ Poisson, Phys. Rev. D {\bf 54}, 5939 (1996).

\bibitem{haris}
    T. Apostolatos, C. Cutler, G.~J. Sussman, and  K.~S. Thorne,
    Phys. Rev. D {\bf 49} 6274 (1994).

\bibitem{BCV}
    A. Buonanno, Y. Chen, and M. Vallisneri, Phys.\ Rev.\ D {\bf 67},  024016 (2003).

\bibitem{MTW}
    C.\ W.\ Misner, K.\ S.\ Thorne, and J.\ A.\ Wheeler, {\it
    Gravitation} (Freeman, San Francisco, 1973), chapter 33.

\bibitem{markovic}
    Markovi\'c, D.\ M.\ 1993, Phys.\ Rev.\ D 48, 4738.

\bibitem{JunkerSchaefer}
    W. Junker and G. Sch\"afer, Mon.\ Not.\ R.\ astr.\ Soc.\
    {\bf 254}, 146 (1992).

\bibitem{Brumberg}
    V. A. Brumberg, {\it Essential Relativistic Celestial Mechanics}
(IOP Publishing, Bristol, 1991).

\bibitem{ryan96}
    F.\ D.\ Ryan, Phys.\ Rev.\ D {\bf 53}, 3064 (1996).

\bibitem{Barker}
    B. M. Barker and R. F. O'Connell, Phys.\ Rev.\ D {\bf 12}, 329 (1975).

\bibitem{Blanchet02} L. Blanchet, G. Faye, B. R. Iyer, and B. Joguet,
Phys.\ Rev.\ D {\bf 65}, 061501 (2002).
 
 \bibitem{Cutler-Kennefick-Poisson}
C.\ Cutler, D.\ Kennefick and E.\ Poisson, Phys. Rev. D
{\bf 50}, 3816 (1994).

\bibitem{cutler_flanagan} C. Cutler, and E.~E. Flanagan, Phys. Rev. D {\bf 49}  2658 (1994).

%\bibitem{Hughes02}
  %      S.\ A.\ Hughes, Mon.\ Not.\ R.\ Astron.\ Soc.\ {\bf 331}, 805 (2002).
%''Untangling the Merger History...''

%\bibitem{Hartl_03}
 %   M.\ D.\ Hartl; gr-qc/0302103.

%\bibitem{Burko_03}
 %   L.\ Burko; gr-qc/0308003.

\bibitem{Kidder_95}
    L. E. Kidder, Phys.\ Rev.\ D {\bf 52}, 821 (1995).

\end{thebibliography}
\end{document}