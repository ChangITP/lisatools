 \documentclass[prd,aps,tightenlines,superscriptaddress,nofootinbib,eqsecnum,amsfonts,amsmath,epsfig]{revtex4} 
\input{epsf} 
\usepackage{thumbpdf}
\usepackage{pdfsync}
\usepackage{graphicx}
\def\be{\begin{equation}}
\def\en{\end{equation}}
\def\bea{\begin{eqnarray}}
\def\ena{\end{eqnarray}}
\def\di{\partial}
\def\bSo{{\bf \hat{S}_1}}
\def\bSt{{\bf \hat{S}_2}}
\def\bL{{\bf \hat{L}_{N}}}
\def\bk{{\bf \hat{k}}}
\def\bp{{\bf p} }


\begin{document}

\title{LMDC: Model for SMBH (comparable masses).}
\author{Stas Babak}
\date{\today}

\maketitle

\section{Models overview}
Currently the following PPn models are available:
\begin{itemize}
\item Circular orbit, non-spinning bodies

Here we know the phase up to 3.5PN order and amplitude up 
to 2.5PN in usual Taylor expansion in velocity or equivalently in
$x = (M\omega)^{2/3}$, where $M = m1+m2$ is total mass and 
$\omega$ is orbital angular frequency (we work in geometrical units:
$G=c=1$). Besides this different resummation techniques could be applied: Pade, EOB (effective one body), see \cite{BCV1} for overview
of various models.

\item Circular orbit, spinning bodies.

Here we know the carrier phase with (almost) the same precision
as non-spinning (3.5PN): namely we have spin contribution only 
formally on 1.5PN level (spin-orbit) and on 2PN (spin-spin). 
PN corrections to leading 1.5PN (spin-orbit) term will be available
 soon. Amplitude corrections are known up to 1.5PN order \cite{Kidder}.
 
 
 \item Eccentric orbit, non-spinning bodies
 
Here we know phase up to 3.5 PN (very recent result) and 
amplitude up to 2PN beyond leading term. Expressions are 
very complicated. Spin effects will be included in the near
future. 
 
\end{itemize}

I want to propose to use only one model for BBH injections
in the first/second parts of LMDC, this is the model for two
spinning bodies in quasi-circular orbit. 

Pros:

We expect that SMBH will be spinning (realism).
We can study effect of spins on the parameter determination.
We can study effect of higher order harmonics on the 
parameter determination. Relatively easy to generate (software).


Cons:

I am not aware of the existing search algorithm for these 
signals with reliable parameter determination. Less well known
as compared to signals from non-spinning binaries. 
Despite the mentioned 'cons', it could have positive effect as 
it could boost research in those directions.


\section{Details of the proposed model}

In first two subsections we define orbital evolution and 
waveform in the source and in radiation frames as defined 
in \cite{FC, Kidder, BCV2}.

We start with defining the notations. The masses of bodies
are defined by $m_1, m_2$. The total mass is $M=m_1+m_2$,
symmetric mass ration $\eta = m_1m_2/M^2$, the reduced 
mass $\mu = m_1m_2/M$ and $\delta m = m_1-m_2$ is mass difference.
 Inspiral is described by the the 
orbital angular frequency $\omega$, the orbital phase $\Phi$,
the direction $\bL \sim \bf{r\times v}$ of the orbital angular momentum, and the two spins ${\bf S_1} =  \chi_1m_1^2 \bSo$
and ${\bf S_2} = \chi_2m_2^2 \bSt$, where $\bSo, \bSt$ are 
unit vectors and $0\le \chi_{1,2} < 1$. Bold fonts denote 
3-d vectors and hats denote the unit vectors. We use geometrical
units $G=c=1$.

\subsection{Orbital evolution}

For orbital angular frequency evolution we can use 
simple Taylor expression with spin-orbital (1.5PN) and
spin-spin (formal 2PN) terms as given in \cite{BCPV}
eqns.(1-7) with $\hat{\theta} = \theta -3/7\lambda = 
\frac{1039}{4620}$.
Or explicitly:

\bea
\frac{d\omega}{dt} = \frac{96}{5}\frac{\eta}{M^2}(M\omega)^{11/3}
\left\{ 1 + 1PN + 1.5PN + SO + 2PN + SS + 2.5PN + 3PN + 3.5PN\right\}
\label{domdt}
\ena
and
\bea
1PN &=& -\frac{743 + 924\eta}{336}(M\omega)^{2/3} \\
1.5PN &=& 4\pi (M\omega) \\
SO &=& -\frac1{12}\sum_{i=1,2}\left[ 
\chi_i\left(\bL.{\bf \hat{S}_i}\right)\left(113\frac{m_i^2}{M^2} + 75\eta\right)
\right](M\omega)\\
2PN &=& \left( \frac{34103}{18144} +\frac{13661}{2016}\eta +\frac{59}{16}\eta^2
\right) (M\omega)^{4/3}\\
\SS &=& -\frac1{48}\eta\chi_1\chi_2\left[ 247(\bSo.\bSt) - 721(\bL.\bSo)(\bL.\bSt)
\right](M\omega)^{4/3}\\
2.5PN &=& -\frac1{672}(4159 + 14532\eta)\pi (M\omega)^{5/3}\\
3PN &=& \left[ \left( \frac{16 447 322 263}{139 708 800} - \frac{1 712}{105}
\gamma_{E} + \frac{16}{3}\pi^2\right) + \left( -\frac{273 811 877}{1 088 640}
+ \frac{451}{48}\pi^2 -\frac{88}{3}\hat{\theta} \right)\eta\right.\nonumber\\
& & \left. + \frac{541}{896}\eta^2 - \frac{5 605}{2 592}\eta^3 - \frac{856}{105}
\ln\left(16(M\omega)^{2/3}\right)\right](M\omega)^{2}\\
3.5PN &=& \left( -\frac{4415}{4032} + \frac{661 775}{12096}\eta + 
\frac{149 789}{3 024}\eta^2\right)\pi (M\omega)^{7/3},
\ena
where $\gamma_{E}$ is Euler's constant. 
Note that transformation $t\to t/M, \omega \to \omega M$,
leads to eliminating total mass. This implies that the waveform
for different total masses can be obtained by simple re-scaling.

For precession of spins and newtonian angular momentum 
${\bf L}_N = \eta M^2(M\omega)^{-1/3}\bL$
we can use eqns.(4.17a-c) in \cite{Kidder} or equivalently
eqns.(8-10) in \cite{BCPV}:

\bea
\frac{d\bSo}{dt} &=& \frac{(M\omega)^2}{2M}\left\{ \eta (M\omega)^{-1/3}
\left(4+3\frac{m_2}{m_1}\right)\bL + \chi_2\frac{m_2^2}{M^2}\left[
\bSt - 3(\bSt.\bL)\bL\right]\right\}\times \bSo \\
\frac{d\bSt}{dt} &=& \frac{(M\omega)^2}{2M}\left\{ \eta (M\omega)^{-1/3}
\left(4+3\frac{m_1}{m_2}\right)\bL + \chi_1\frac{m_1^2}{M^2}\left[
\bSo - 3(\bSo.\bL)\bL\right]\right\}\times \bSt\\
\frac{d\bL}{dt} &=& {\bf V_{L_N} }\times\bL \equiv \frac{(M\omega)^2}{2M}
\left\{ \left(4 + 3\frac{m_2}{m_1}\right)\chi_1\frac{m_1^2}{M^2}\bSo +
\left(4 + 3\frac{m_1}{m_2}\right)\chi_2\frac{m_2^2}{M^2}\bSt- \right.
\nonumber\\
&& \left. 3(M\omega)^{1/3}\eta\chi_1\chi_2\left[ (\bSt.\bL)\bSo +
(\bSo.\bL)\bSt \right] \right\}\times\bL.\label{dLn}
\ena
Alternatively, instead of the last equation we can use 
$$
\bL = \{ \sin(i)\cos(\alpha), \sin(i)\sin(\alpha), \cos(i) \}
$$ 
which leads to evolution of $\alpha, i$:
\bea
\frac{di}{dt} &=& V_{L_N}^{y} \cos(\alpha) - V_{L_N}^{x}\sin(\alpha)\\
\frac{d\alpha}{dt} &=& V_{L_N}^z - \frac{\cos(i)}{\sin(i)}\left[ 
V_{L_N}^{x} \cos(\alpha) + V_{L_N}^{y}\sin(\alpha) \right]
\ena
Here we have used condition that $\sin(i) \neq 0$, and ${\bf V_{L_N} } = 
\{ V_{L_N}^{x}, V_{L_N}^{y}, V_{L_N}^{z}\}$.
In the implementation of this scheme (described below), I integrate  
equations for $(i, \alpha)$ and eqn.(\ref{dLn}) and check consistency
(error of integration).

Finally, for the phase evolution we have:
\bea
\frac{d\Phi}{dt} = \omega - \frac{d\alpha}{dt}\cos(i).
\ena 

Besides masses $m_1, m_2$ and amplitude of spins $\chi_1, \chi_2$,
one has to specify the initial conditions. Initial data for the differential 
equations specified at $t=t_0$, those are initial directions of spins:
$\bSo(t_0), \bSt(t_0)$, initial direction of the orbital angular momentum
$i(t_0), \alpha(t_0)$, initial frequency $\omega_0 = \omega(t_0)$ and
initial phase $\Phi_0 = \Phi(t_0)$. 
These values are defined in source coordinate frame (see nice figure in \cite{BCV2}).

Termination point (follow \cite{BCV2}) is defined as
$$
min_{\omega}\left\{ \dot{\omega}= 0; \frac{dE_{3PN}}{d\omega} = 0\right\}
$$
and referred to as MECO.
One can write $\frac{dE_{3PN}}{d\omega} = 0$ explicitly using equations above
and 
$$
\frac{d}{d\omega} = \frac{1}{\dot{\omega}} \frac{d}{dt}
$$
when we need to differentiate vectors. I use $\dot{\omega}$ in the above
equation only up to 1PN order, neglecting others. Following this prescription 
and neglecting term cubic in spins: $O(S^3)$ one can arrive at the following:

\bea
\frac1{M} \frac{dE_{3PN}}{d\omega} = -\frac{\mu}{3}(M\omega)^{-1/3}\left\{
1 - \frac{9+\eta}{6}(M\omega)^{2/3} + \frac{20}{3M^2}(\bL.{\bf S_{eff}})(M\omega)+
\right. \nonumber\\
\left. \frac5{64}\left( 1 - \frac3{8}\eta\right) (M\omega)^{1/3} \frac{\delta m}{M}
\chi_1\chi_2\left[ 1 + \frac{743 + 924\eta}{336}(M\omega)^{2/3}\right]
(\bSo\times\bSt)\bL \right. \nonumber \\
\left. \frac{1}{8}(-81 + 57\eta -\eta^2)(M\omega)^{4/3} + \frac{3\eta}{4}
\chi_1\chi_2 \left[ (\bSo.\bSt) - 3(\bL.\bSo)(\bL.\bSt) \right](M\omega)^{4/3}
\right. \nonumber \\
\left. 4\left[ -\frac{675}{64} + \left( \frac{34445}{576} - \frac{205}{96}\pi^2
\right)\eta -\frac{155}{96}\eta^2 - \frac{35}{5184}\eta^3 \right](M\omega)^{2}
\right\},
\ena 
where 
$$
{\bf S_{eff}} = \left( 1 + \frac3{4}\frac{m_2}{m_1} \right) {\bf S_1}+ 
\left( 1 + \frac3{4}\frac{m_1}{m_2} \right) {\bf S_2}
$$
For non-spinning binaries $\omega_{MECO} = 0.129M^{-1}$, compared to the LSO
for test mass case ($\eta \to 0$), $\omega_{LSO} = 0.068M^{-1}$. I find that 
MECO gives quite high ending frequency (almost twice LSO).


\subsection{Gravitational Waveform}

The waveform is defined in \cite{Kidder}, eqn.(4.8) up to 
1.5PN. and it is given explicitly in the AppendixB 
up to 1PN. In the radiation coordinate frame (see \cite{FC} eqns.(3.1-3.3),
\cite{Kidder} eqns.(4.22a-c), \cite{BCV2} eqns.(21-23) )
waveform is given as
\begin{eqnarray}
h_{TT}^{ij} = h_+T^{ij}_+ + h_{\times}T^{ij}_{\times}\\
{\bf T}_{\times} = {\bf e}_x^R\otimes{\bf e}_x^R - {\bf e}_y^R\otimes{\bf e}_y^R\\
{\bf T}_{\times} = {\bf e}_x^R\otimes{\bf e}_y^R + {\bf e}_y^R\otimes{\bf e}_x^R\\
h_+ = \frac1{2}h^{ij}(T_+)_{ij}, \,\,\, 
h_{\times} = \frac1{2}h^{ij}(T_{\times})_{ij}
\end{eqnarray}

Radiative system introduces one more angle $\theta$ which
is defined as inclination of direction to the observer $\hat{\bf N}$
to the total angular momentum at $t_0$\footnote{Another angle, $\phi$, defining 
direction to the observer, can always
be put to zero by rotating coordinate frame around z-axis}. 

Finally waveform (plus and cross polarizations in the radiation
coordinate frame) can be written as 

\begin{equation}
h_{+,\times} = \frac{2\mu}{D}\frac{M}{r}\left[ 
Q_{+,\times}  + \left(\frac{M}{r}\right)^{1/2}Q^{1}_{+,\times} +
\left(\frac{M}{r}\right)Q^{2}_{+,\times}
 + {\mathcal O}\left(\frac{M}{r}\right)^{3/2}\right]
\end{equation}
where expressions for $Q^{1}_{+,\times} = P^{0.5}_{+,\times},\;\; Q^{2}_{+,\times} = PQ_{+,\times}$ are given in AppendixB of \cite{Kidder}. 
In order to show the conventions used here we will give the formulae explicitly
(should be useful if one wants to use different coordinate frame or wants to
check our result). 
\bea
Q_{+,\times} = \frac1{2} Q^{ij}(T_{+,\times})_{ij}; \;\;\;
Q^1_{+,\times} = \frac1{2} (Q^1)^{ij}(T_{+,\times})_{ij}; \;\;\;
Q^2_{+,\times} = \frac1{2} (Q^2)^{ij}(T_{+,\times})_{ij}.
\ena
where
\bea
Q^{ij} &=&  2(\lambda^i\lambda^j - n^in^j), \\
(Q^1)^{ij} &=& \frac{\delta m}{m}\left\{ 6({\bf \hat{N}.n})n^{(i}\lambda^{j)} +
({\bf \hat{N}.\lambda})(n^in^j - 2\lambda^i\lambda^j) \right\}\\
(Q^2)^{ij} &=& \frac2{3}(1-3\eta)\left\{ ({\bf \hat{N}.n})^2(5n^in^j -
7\lambda^i\lambda^j) - 16 ({\bf \hat{N}.n})({\bf \hat{N}.\lambda})n^{(i}\lambda^{j)}
 ({\bf \hat{N}.\lambda})^2(3\lambda^i\lambda^j - n^in^j)\right\} \nonumber \\
 & &\frac1{3}(19-3\eta)(n^in^j - \lambda^i\lambda^j) + \frac2{M^2}n^{(i}
 ({\bf \Delta \times \hat{N}})^{j)},
\ena 
where ${\bf \lambda}$ is unit vector along relative velocity ${\bf v}$, 
${\bf n}$ is unit vector along the separation vecotr of binary ${\bf r}$,
${\bf \hat{N}}$ is the unit vector in the direction of the observer (SSB) and 
$$
\frac{{\bf \Delta}}{M^2} = \chi_2\frac{m_2}{M}\bSt - \chi_1\frac{m_1}{M}\bSo. 
$$
In the source coordinate frame:
\bea
{\bf n} &=& \{ -\sin(\alpha)\sin(\Phi)-\cos(\alpha)\cos(i)\cos(\Phi), 
-\cos(\alpha)\sin(\Phi)-\sin(\alpha)\cos(i)\cos(\Phi), \sin(i)\cos(\Phi) \}\\
{\bf \lambda} &=& \{  \sin(\alpha)\sin(\Phi)-\cos(\alpha)\cos(i)\cos(\Phi), 
-\cos(\alpha)\sin(\Phi)-\sin(\alpha)\cos(i)\cos(\Phi), \sin(i)\cos(\Phi) \},\\
{\bf \hat{N}} &=& \{ \sin(\theta), 0, \cos(\theta) \}.
\ena
The radiation frame related to source frame in the following way
\bea
{\bf \hat{e}_x^R} &=& {\bf \hat{e}_x^S}\cos(\theta) - {\bf \hat{e}_z^S}\sin(\theta)\\
{\bf \hat{e}_y^R} &=& {\bf \hat{e}_y^S}\\
{\bf \hat{e}_z^R} &=& {\bf \hat{e}_x^S}\sin(\theta) + {\bf \hat{e}_z^S}\cos(\theta) = 
{\bf \hat{N}}.
\ena
So that 
\bea
{\bf T_+} &=& {\bf \hat{e}_x^S}\otimes{\bf \hat{e}_x^S}\cos^2(\theta) - 
{\bf \hat{e}_x^S}\otimes{\bf \hat{e}_z^S}\sin(\theta)\cos(\theta) - 
{\bf \hat{e}_z^S}\otimes {\bf \hat{e}_x^S}\sin(\theta)\cos(\theta) -
{\bf \hat{e}_y^S}\otimes{\bf \hat{e}_y^S} +
{\bf \hat{e}_y^S}\otimes{\bf \hat{e}_y^S}\sin^2(\theta)\\
{\bf T_x} &=& ({\bf \hat{e}_x^S}\otimes{\bf \hat{e}_y^S} +
 {\bf \hat{e}_y^S}\otimes{\bf \hat{e}_x^S})\cos(\theta) - 
 ({\bf \hat{e}_y^S}\otimes{\bf \hat{e}_z^S} +
 {\bf \hat{e}_z^S}\otimes{\bf \hat{e}_y^S})\sin(\theta)
\ena
Using these relationships we have for $Q$s (we give expression for "+" polarization
only, "x" polarization can be obtained by replacing "+" with "x"):
\bea
Q_+ &=& -2(C_+ \cos(2\Phi) + S_+\sin(2\Phi))\\
Q^1_+ &=& \frac1{4}\frac{\delta m}{M}\left\{ 9(aS_+ + bC_+)\cos(3\Phi) +
9(bS_+ -aC_+)\sin(3\Phi) + \right. \nonumber\\ 
& & \left.(3aS_+ -3bC_+ -2bK+)\cos(\Phi) -
(3bS_+ + 3aC_+ -2aK_+)\sin(\Phi)\right\}\\
Q^2_+ &=& \frac8{3}(1-3\eta)\left\{ [(a^2-b^2)C_+ -2abS_+]\cos(4\Phi)
+[(a^2-b^2)S_+ + 2abC_+]\sin(4\Phi)\right\} \nonumber \\
& &+ \frac1{3}\left\{ 2(1-3\eta)[(b^2-a^2)K_+ + 2(a^2+b^2)C_+] +
(19 -3\eta)C_+  \right\}\cos(2\Phi) + \nonumber \\
& &+ \frac1{3}\left\{ 4(1-3\eta)[(a^2+b^2)S_+ - abK_+]  + (19-3\eta)S_+ 
\right\}\sin(2\Phi) + DC_+\cos(\Phi) + DS_+\sin(\Phi). \label{Q2}
\ena
where 
\bea
C_+ &=& \frac1{2}\cos^2(\theta)\left( \sin^2(\alpha) - \cos^2(i)\cos^2(\alpha)\right)
+ \frac1{2}\left(\cos^2(i)\sin^2(\alpha) - \cos^2(\alpha)\right) -
\nonumber \\
& &\frac1{2}\sin^2(\theta)\sin^2(i) - \frac1{4}\sin(2\theta)\sin(2i)\cos(\alpha),\\
C_{\times} &=& -\frac1{2}\cos(\theta)\sin(2\alpha)\left( 1+\cos^2(i) \right)
-\frac1{2}\sin(\theta)\sin(2i)\sin(\alpha),\\
S_+ &=& \frac1{2}\left(1 + \cos^2(\theta)\right)\cos(i)\sin(2\alpha) + 
\frac1{2}\sin(2\theta)\sin(i)\sin(\alpha),\\
S_{\times}&=& -\cos(\theta)\cos(i)\cos(2\alpha) - \sin(\theta)\sin(i)\cos(\alpha)\\
K_+ &=& \frac1{2}\cos^2(\theta)\left( \sin^2(\alpha) + \cos^2(i)\cos^2(\alpha)\right)
- \frac1{2}\left(\cos^2(i)\sin^2(\alpha) + \cos^2(\alpha)\right) +
\nonumber \\
& &\frac1{2}\sin^2(\theta)\sin^2(i) + \frac1{4}\sin(2\theta)\sin(2i)\cos(\alpha),\\
K_{\times} &=& -\frac1{2}\cos(\theta)\sin(2\alpha)\sin^2(i) 
+\frac1{2}\sin(\theta)\sin(2i)\sin(\alpha),\\
DC_+ &=& -\frac1{M^2}\left[ \Delta^{y}\sin(\alpha)\cos(\theta) + d\cos(\alpha)\right],\\
DS_+ &=& -\frac1{M^2}\left[ c\Delta^y - d\cos(i)\sin(\alpha)\right], \\
DC_{\times} &=& \frac1{M^2}\left[ \Delta^y\cos(\alpha) -d\cos(\theta)\sin(\alpha)
\right],\\
DS_{\times} &=& \frac1{M^2}\left[ -\Delta^y\cos(i)\sin(\alpha) + cd \right],\\
a &=& -\sin(\theta)\sin(\alpha), \\
b &=& \cos(\theta)\sin(i) -\sin(\theta)\cos(i)\cos(\alpha),\\
c &=& \cos(\theta)\cos(i)\cos(\alpha) + \sin(i)\sin(\theta),\\
d &=& \Delta^z\sin(\theta) - \Delta^x\cos(\theta).
\ena
Since we integrate orbital angular frequency we need to relate 
$M/r$ to $(M\omega)$ (with sufficient for our purpose accuracy):
$$
\frac{M}{r} \approx (M\omega)^{2/3}\left[ 1 + \frac1{3}(3-\eta)(M\omega)^{2/3}\right]
$$
Taking the above into account we can rewrite $h_{+,\times}$ as follows:
\bea
h_{+,\times} = \frac{2\mu}{D}(M\omega)^{2/3}\left[ 
Q_{+,\times}  + (M\omega)^{1/3}Q^{1}_{+,\times} +
(M\omega)^{2/3}\left(Q^{2}_{+,\times} + \frac1{3}(3-\eta)Q_{+,\times}\right)
 + {\mathcal O}(M\omega)\right]\label{hom}
\ena
The last term in (\ref{hom}) yields the following change in (\ref{Q2}), instead of
$(19 - 3\eta)C_+$, $(19 - 3\eta)S_+$ we have $(13-\eta)C_+$, $(13-\eta)S_+$
correspondingly. 

In order to generate waveform using the inspiralling trajectory  
described in the previous subsection one need to specify 
distance between observer and the source, and the direction 
to the observer in the source frame $\theta, \phi=0$.


\section{Reduction to non-spinning case}

Reducing from spinning to non-spinning case can be done pretty easily
We start with orbital evolution 

\subsection{Non-spinning case: orbital evolution}

The orbital evolution is described by only one differential equation which can
be integrated analytically. This is the equation for the orbital frequency
(\ref{domdt}) with $SO=SS=0$. 
The analytic expressions for frequency and phase are given in \cite{Blanchet}
we also give them here. First introduce 
\bea
\tau = \frac{\eta}{5M}(t_c - t)\\
x = v^2 = (M\omega)^{2/3}
\ena
Then 
\bea
(M\omega)^{2/3} &=& \frac1{4}\tau^{-1/4}\left\{ 1 + \left(\frac{743}{4032} + \frac{11}{48}\eta\right)
\tau^{-1/4} - \frac1{5}\pi \tau^{-3/8} + \right.\nonumber \\
& &\left. \left( \frac{19583}{254016} + \frac{24401}{193536}\eta + \frac{31}{288}\eta^2\right)
\tau^{-1/2} + \left(-\frac{11891}{53760} + \frac{109}{1920}\eta\right)\pi \tau^{-5/8} -
\right. \nonumber \\
& & \left. \left[ -\frac{10052469856691}{6008596070400} + \frac1{6}\pi^2 + 
\frac{107}{420}\gamma_E - \frac{107}{3360}\ln\left(\frac{\tau}{256}\right) 
+ \left(\frac{15335597827}{3901685760} - \frac{451}{3072}\pi^2 - 
\frac{77}{72}\lambda + \frac{11}{24}\theta \right)\eta - \right. \right. \nonumber \\
& & \left. \left. \frac{15211}{442368}\eta^2 + \frac{25565}{331776}\eta^3 \right]
\tau^{-3/4} + \left(-\frac{113868647}{433520640} - \frac{31821}{143360}\eta + 
\frac{294941}{3870720}\eta^2\right)\pi\tau^{-7/8}  \right\}
\ena
where $\lambda = -\frac{1987}{3080}, \theta=-\frac{11831}{9240}$.
Or taking the power $3/2$:
\bea
M\omega &=& \frac1{8} \tau^{-3/8}\left\{ 1 + \left( \frac{11}{32}\eta + \frac{743}{2688}\right) 
\tau^{-1/4} - \frac{3}{10}\pi\tau^{-3/8} 
+ \left(\frac{1855099}{14450688} + \frac{371}{2048}\eta^2 + \frac{56975}{258048}\eta\right)
\tau^{-1/2} + \right. \nonumber \\
& &\left.\left( \frac{13}{256}\eta- \frac{7729}{21504}\right)\pi \tau^{-5/8} +
\left[ \frac{235925}{176472}\eta^3 - \frac{30913}{1835008}\eta^2 + \left(-\frac{451}{2048} \pi^2+ 
\frac{25302017977}{4161798144}\right) eta -\right. \right. \nonumber \\
& & \left. \left.
\frac{720817631400877}{288412611379200} + \frac{53}{200}\pi^2 + \frac{107}{280}\gamma_E - 
\frac{107}{2240}\ln\left(\frac{\tau}{256}\right) \right]\tau^{-3/4} +
\right. \nonumber \\
& & \left. \left(\frac{141769}{1290240}\eta^2- \frac{188516689}{433520640} -\frac{97765}{258048}\eta\right) 
\pi\tau^{-7/8}
\right\}
\label{omNosp}
\ena

The termination frequency is defined by MECO \cite{BCV2}: $\omega_{MECO} = 0.129M^{-1}$ or 
by condition $\dot{\omega}=0$.

The orbital phase could be expressed in terms of orbital frequency
\bea
\Phi &=& -\frac1{32\eta}(M\omega)^{-5/3}\left\{ 1 + \left( \frac{3715}{1008} + \frac{55}{12}\eta\right)
(M\omega)^{2/3} - 10\pi(M\omega) + \right. \nonumber \\
& & \left. \left( \frac{15293365}{1016064} + \frac{27145}{1008}\eta + \frac{3085}{144}\eta^2\right)
(M\omega)^{4/3} + \left( \frac{38645}{1344} - \frac{65}{16}\eta \right)\pi\ln\left( \frac{\omega}{\omega_0}\right) (M\omega)^{5/3} \right. \nonumber \\
& &  \left. \left[ \frac{12348611926451}{18776862720} -\frac{160}{3}\pi^2 - \frac{1712}{21}\gamma_E
- \frac{856}{21}\ln\left[ 16(m\omega)^{2/3}\right] + \left( -\frac{15335597827}{12192768} +
\frac{2255}{48}\pi^2 + \frac{3080}{9}\lambda - \right.\right.\right. \nonumber \\
& & \left.\left.\left.  \frac{440}{3}\theta\right)\eta - \frac{76055}{6912}\eta^2 - 
\frac{127825}{5184}\eta^3\right](M\omega)^{2} + \left( \frac{77096675}{2032128} + 
\frac{378515}{12096}\eta - \frac{74045}{6048}\eta^2\right)\pi(M\omega)^{7/3} 
\right\}
\ena

This is waveform "TaylorT3" in classification described in \cite{DIS} if one substitutes 
orbital angular frequency (\ref{omNosp}) in the expression for phase above.

\subsection{Gravitational waveform: non-spinning case}

The reduction to non-spinning case can be easily done by using (\ref{hom})
and setting 
$$
\alpha =0,\;\;\; i = 0.
$$
Together with $\chi_1=\chi_2=0$.
This reduces the long expression to relatively short\footnote{Note that here $\theta$ angle correspond
to $i$ used in \cite{FC}}:
\bea
C_{+} = -\frac1{2}(1+ \cos^2(\theta)), \;\;\; C_{\times} = 0\\
S_{+} = 0,\;\;\; S_{\times} = -\cos(\theta)\\
K_{+} = -\frac1{2}\sin^2(\theta),\;\;\; K_{\times} = 0\\
DC_{+}=DC_{\times}=DS_{+}=DS_{\times} = 0\\
a=0,\;\;\; d=0,\;\;\; b=-\sin(\theta),\;\;\; c=\cos(\theta). 
\ena
Substituting these expressions into (\ref{Q2}) we get:

\bea
Q_{+} &=& (1+\cos^2(\theta))\cos(2\Phi)\\
Q_{\times} &=& 2\cos(\theta)\sin(2\Phi)\\
Q^{(1)}_{+} &=& \frac1{8}\frac{\delta m}{M}\sin(\theta)
\left[ 9(1+\cos^2(\theta))\cos(3\Phi) - (5+\cos^2(\theta))\cos(\Phi) \right]\\
Q^{(1)}_{\times} &=& \frac3{4} \frac{\delta m}{M}\sin(\theta)
\cos(\theta)\left[ 3\sin(3\Phi) - \sin(\Phi) \right]\\
\frac1{3}(3-\eta)Q_{+} + Q^{(2)}_{+} &=& \frac4{3} (1-3\eta)
(1-\cos^4(\theta))\cos(4\Phi) + \nonumber \\
& &\frac1{6}\left[ -19 + 19\eta -(9 + 11\eta)\cos^2(\theta) + (1-3\eta)\cos^4(\theta)\right] \cos(2\Phi)\\
\frac1{3}(3-\eta)Q_{\times} + Q^{(2)}_{\times} &=&
\frac8{3}(1-3\eta)(1-\cos^2(\theta))\cos(\theta)\sin(4\Phi) - 
\nonumber \\
& &\frac1{3}\left[ 17  -13\eta -4(1-3\eta)\cos^2(\theta)
 \right]\cos(\theta)\sin(2\Phi)
\ena
which is in agreement with result derived by number of authors.




\section{Transformation to SSB}
\label{SSB}

Transformation to Solar-System-Baricentric ecliptic coordinates 
is given by $O_1$ rotation matrix given in \cite{KTV} eqn.7-9.
Transformation to the LISA frame is given by eqn.6,9 \cite{KTV}.
For completeness we will write it here explicitly. 
At the origin of the SSB frame, the transverse-traceless metric perturbation due 
to a source located at ecliptic latitude $\beta$ and longitude $\lambda$ can be written as
\be
H(t) = O_1 H^S(t)O_1^{-1}
\en
where the metric perturbation in the source frame is taken to be
\be
H^S(t) = \left( \begin{array}{ccc}
h_{+}(t) & h_{\times}(t) & 0 \\
h_{\times}(t) & -h_{+}(t) & 0 \\
0 & 0 & 0 \end{array} \right),
\en
The two polarizations are given by eqn.(\ref{hom}) and the rotation matrix
$O_1$:
\be
O_1 =\left( \begin{array}{ccc}
\sin(\lambda)\cos(\psi) - \cos(\lambda)\sin(\beta)\sin(\psi) &  
-\sin(\lambda)\sin(\psi) - \cos(\lambda)\sin(\beta)\cos(\psi) & 
-\cos(\lambda)\cos(\beta) \\  
-\cos(\lambda)\cos(\psi)  - \sin(\lambda)\sin(\beta)\sin(\psi) &
\cos(\lambda)\sin(\psi) - \sin(\lambda)\sin(\beta)\cos(\psi) &
 -\sin(\lambda)\cos(\beta) \\  
\cos(\beta)\sin(\psi) &  \cos(\beta)\cos(\psi) & -\sin(\beta)  
 \end{array} 
\right).
\en 
These equations describe the plane waves, which are
propagating from a source located in the direction
\be
{\bf \hat{k}} = \{ \cos(\lambda)\cos(\beta), \sin(\lambda)\cos(\beta), 
\sin(\beta)\}.
\en
The polarization angle $\psi$ encodes a rotation around the direction of wave propagation, $-{\bf \hat{k}}$, setting the convention used to define the two polarizations, "$+$" and "$\times$".

So in order to describe the gravitational wave signal in SBB  one requires 
3 more parameters determining the position of the source and the polarization angle
\begin{equation}
\lambda, \beta, \psi
\end{equation}
%Gravitational wave responses are given by eqns.(12-14) \cite{KTV}.
%Taking care of Dopler shift in phase...
%\begin{eqnarray}
%\int f(\tau) d\tau &=& \left[ \tau = t + {\bf k}{\bf p }_i \right] = 
%\int f(t + {\bf k}{\bf p }_i) \left(1 + {\bf k}\frac{{\bf p }_i}
%{dt}\right) dt \approx \int \left[ f(t) + \dot{f}(t)* {\bf k}{\bf p }_i\right]\left(1 
%+ {\bf k}\frac{{\bf p }_i}
%{dt}\right) dt \approx \nonumber \\
%& &\int  f(t) dt + f(t) {\bf k}{\bf p }_i
%\end{eqnarray}
 
%Check if the above approximation is reasonable and expand ${\bf p}_i$



\section{Parameter distribution}

Here I will try to define the distribution of the parameter for injections.
\begin{itemize}

\item {\it Masses}. Individual masses are uniformly distributed in the range
$[10^5 - 10^{7}]M_{\odot}$. For first challenges we can reduce upper bound to
$10^6M_{\odot}$ as this will increase duration of the waveform in the LISA band.

\item {\it Spins}. Initial conditions for spins. 
For spinning binaries $\chi_1,\chi_2$ are uniformly distributed
on $[0,1]$, $\hat{S}_i^z$ are uniformly distributed on $[-1,1]$, and angles
$\phi_{S_i}$ are uniformly distributed on $[0,2\pi]$

\item {\it Initial orbit} Initial orbit is defined by initial direction
of the orbital angular momentum $\bL$, initial phase and initial orbital frequency.
$\bL^z = \cos(i)$ is uniformly distributed on $[-1,1]$, $\alpha$ and $\Phi_0$ are uniformly
distributed on $[0,2\pi]$. The initial orbital angular frequency $\omega_0$ 
will be defined by the length of the duration of the signal $t_{cut}$. User should define 
$\Delta t = t_c - t_{cut}$ (say 1 week before $t_c$).

\item {\it Direction to the observer}. As mentioned in the text $\phi=0$ and
$\cos(\theta)$ is distributed uniformly on $[-1,1]$

\item {\it Distance to the observer}. Distance to the source should be defined
by SNR for a given direction. Here there is an option: 1. try to estimate actual
SNR for a given parameters 2. estimate SNR for non-spinning binary for the same
location on the sky, the same masses 3. Restricted SPA waveform for moving or
static LISA

\item {\it Location on the sky in SSB}. In SSB we need to set the direction to the source
and polarization angle: $\sin(\beta)$ is distributed uniformly on $[-1,1]$, and
$\lambda, \psi$ are uniformly distributed on $[0,2\pi]$

\end{itemize}

\section{Implementation}

The computation of GW from the spinning binary (in the source frame)
is implemented as  C++ class. It has the following methods:
\begin{itemize}
\item {\tt SpinBBHWaveform(float mass1, float mass2)}, Constructor, the
input are two masses in units of solar mass.

\item {\tt SetInitialSpins(float chi1, float S1z0, float phiS10, float chi2, float S2z0,
		   float phiS20)}. Here user specifies the initial value for spins.
The spins are specified by its amplitude {\tt chi1, chi2}, and the orientation
is defined $\bSo = \{ \sqrt{1-(S1z0)^2}\cos(phiS10), \sqrt{1-(S1z0)^2}\sin(phiS10),
S1z0\}$ (similar for $\bSt$). 
This parametrization was chosen as the most convenient for generating the 
random signal, since {\tt S1z0, S2z0} are distributed uniformly on $[-1,1]$ and
{\tt phiS10, phiS20} are distributed uniformly on $[0, 2\pi)$. 

\item {\tt void SetInitialOrbit(float omega0, float phi0, float iota0, float alpha0)}
Here user specifies initial orbit: initial frequency {\tt omega0}, initial
 phase {\tt phi0}, and initial direction of the orbital angular momentum 
 {\tt iota0, alpha0} (corresponding to $i,\alpha$). 
 
\item {\tt void ComputeInspiral(float timeStep)}. This method integrates 
equations of motion and as a result produces the inspiralling trajectory.
User must specify sampling step {\tt timeStep} in seconds.

\item {\tt void SetObserver(float thetaD, double D)}. Here user specifies direction
{\tt thetaD} and distance {\tt D} (in pc) to the observer.

\item {\tt void ComputeWaveform(int order,  Matrix<double>\& hPlus, Matrix<double>\& hCross)}. This method computes the waveform $h_+, h_{\times}$ for observer 
described in {\tt SetObserver} and trajectory computed in {\tt ComputeInspiral}.
This can be called multiple number of times for different observers without 
re-computing inspiral. The parameter {\tt order} allows user to get only leading
order amplitude (0), only 0.5PN amplitude (1), only 1PN amplitude or all of them
together (default).

\end{itemize}

In the figure below one can see the example of the waveform.

As I have mentioned above the MECO condition gives quite high frequency.
I have used the condition $\frac1{M} |dE/d\omega| < 3.5\times 10^{-6}$ to 
terminate inspiral.



\begin{figure}[ht]
%\includegraphics[height=8cm,width=14cm]{SpinBBHWave.pdf}
\caption{Gravitational waveform in the source frame for the 
following parameters: $m_1 = 2.0M_{\odot}, m_2=4.0M_{\odot}, 
\chi_1 = 0.7, \chi_2=0.9, \hat{S}_1^z(t_0) = 0.4, \hat{S}_2^z(t_0) = 0.7,
\phi_{S1}(t_0) = \pi/7, \phi_{S2}(t_0) = \pi/3.5, \omega(t_0) = 40\pi, 
\Phi(t_0)=0, i(t_0) = \pi/7, \alpha(t_0) = \pi/3, \theta=\pi/2.2, D=10^6(pc)$}
\label{fig:waveforms}
\end{figure}

\section{Model of the signal for the first mock data challenge}

Here we summarize the model which will be used for the first 
LISA mock data challenge. It is a restricted non-spinning 
model with the phase up to 2PN.

The orbital evolution is presented by orbital angular frequency 
and orbital phase up to 2PN order:

\bea
M\omega &=& \frac1{8} \tau^{-3/8}\left\{ 1 + \left( \frac{11}{32}\eta + \frac{743}{2688}\right) 
\tau^{-1/4} - \frac{3}{10}\pi\tau^{-3/8} 
+ \left(\frac{1855099}{14450688} + \frac{371}{2048}\eta^2 + \frac{56975}{258048}\eta\right)
\tau^{-1/2} 
\right\}
\label{fr}
\ena
And the phase is described by:
\bea
\Phi &=& -\frac1{32\eta}(M\omega)^{-5/3}\left\{ 1 + \left( \frac{3715}{1008} + \frac{55}{12}\eta\right)
(M\omega)^{2/3} - 10\pi(M\omega) +  \left( \frac{15293365}{1016064} + \frac{27145}{1008}\eta + \frac{3085}{144}\eta^2\right)
(M\omega)^{4/3}\right\}
\ena
The evolution is terminated by condition $r=6M$ which corresponds to 
$$
\omega_{LSO} = 0.068(M_{\odot}/M)Hz
$$
unless user wants to terminate it $\Delta t$ sec before the coalescence time $t_c$.

Waveform is described by two polarizations in the 
radiation frame:

\bea
h_{+} &=& \frac{2\mu}{D}(M\omega)^{2/3}(1+\cos^2(\theta))\cos(2\Phi)
\label{hp}\\
h_{\times} &=& \frac{4\mu}{D}(M\omega)^{2/3}\cos(\theta)\sin(2\Phi)
\label{hc}
\ena

The transformation to the SSB system is given in the Section~\ref{SSB}. This is finalizes the time domain description of the 
signal which will be an input to the simulator. 

\subsection{Stationary phase approximation}

Here we derive the fourier image of the time-domain signal 
described above using stationary phase approximation (SPA).

The signal in the radiation frame is given by eqns.(\ref{hp},
\ref{hc}), this signal has to be passed through a particular 
transfer function to get a TDI observable, this can be presented in a rather symbolic form as follows
\bea
h^{I} = \sum_{j=1}^{3}\sum_k A_{k,j}^{I}(t)\left[ 
F^{+}_j(t)h_{+}(t-t_k) + F^{\times}_{j}(t)h_{\times}(t-t_k)
\right]
\ena
where index $I$ corresponds to a different TDI combination $\alpha_i$,
$\bar{A}, \bar{E}, \bar{T},....$; $t_k$ are delays and $A_{k,j}(t)$
 scales an amplitude of each delayed waveform. The antenna pattern 
if given as 
\bea
F_j^{+} &=& u_j(t)\cos(2\psi) + v_j(t)\sin(2\psi),\\
F_j^{\times} &=& v_j(t)\cos(2\psi) - u_j(t)\sin(2\psi)
\ena
where $\psi$ is polarization angle introduced in Section~\ref{SSB}. 
The functions $u_j(t), v_j(t)$ depend on the rotation matrix $O_1$
and on LISA-to-SSB transformation, thus, they depend on time through  motion of the guiding center and cartwheelng motion of the spacecrafts, and on the position of the source in the sky,
given by the ecliptic coordinates $\beta,\lambda$. The explicit 
expressions for those functions are given in \cite{KTV}, eqns.(27-39). 
Following \cite{Cutler} introduce polarization phase $\Phi_p^j(t)$:

\bea
F^{+}_j(t)h_{+}(t-t_k) + F^{\times}_{j}(t)h_{\times}(t-t_k) =
%(M\omega)^{2/3} \left[ F^{+}_j(t)h^{+}_0\cos(2\Phi(t-t_k)) + 
%F^{\times}_{j}(t)h^{\times}_0\sin(2\Phi(t-t_k))\right] =\\
\mathcal{A}^j(t)(M\omega)^{2/3}\cos(2\Phi(t-t_k) -\Phi_p^j(t))
\ena
where
\bea
\mathcal{A}^j(t) &=& \frac{2\mu}{D} \sqrt{(F^{+}_j(t)h^{+}_0)^2 + 
(F^{\times}_{j}(t)h^{\times}_0)^2},\\
\tan(\Phi_p^j) &=& \frac{F^{\times}_{j}(t)h^{\times}_0}
{F^{+}_{j}(t)h^{+}_0}\\
h^{\times}_0 &=& 2\cos(\theta),\;\;\; 
h^{+}_0 = \frac1{2}(1 + \cos^2(\theta))
\ena
As a next step we expand the phase around $t$:
\be
\Phi(t - t_k) \approx \Phi(t) - \omega t_k +....
\en
here we neglected second order Doppler corrections (see \cite{Cutler}
for justification).
It also convenient to represent $\cos$ in exponential form:
\be
\cos(2\Phi(t-t_k) - \Phi_p^j(t)) = \frac1{2}
e^{-i\left[ 2\Phi(t) - 2\omega t_k  - \Phi_p^j(t)\right]} + c.c.
\en
where $c.c.$ stands for complex conjugate, which we will omit 
in further expressions (I assume that we mainly interested in 
positive frequencies).
Here we can identify the Doppler phase $\Phi^k_D(t) = 2\omega t_k$.
The delays can be further expanded according to:
\bea
t_j = \omega R \cos(\beta) \cos(\Omega t + \eta_0 -\lambda) +
\bk(O_2(t)\bp_j^L) - m_jL
\ena
where $R=1au$, $\eta_0$ is initial phase of the LISA's guiding 
center, $O_2$ is the LISA-to-SSB rotation matrix (see \cite{KTV}),
$\bp^L_j$ is the vector connecting the guiding center and $j$-th spacecraft and $m_j$ is integer coming from a particular TDI combination. Note that $\omega \bk(O_2(t)\bp_j^L) \sim \omega L$.

Put all pieces together:
\bea
h^I = \Lambda^I(t) (M\omega)^{2/3}e^{-i(2\Phi(t))},\\
\Lambda^I(t) = \sum_{j=1}^{3}\sum_k \frac1{2}
A_{k,j}^I(t)\mathcal{A}^j(t)e^{i(\Phi_D^k(t) + \Phi_p^j(t))}
\ena
Since $\Lambda(t)$ varies on the timescale $1yr >> 2\pi/\omega$
we can apply SPA to evaluate the waveform in the frequency domain:

\bea
\tilde{h}^I_{spa} = \Lambda(t_f) (M\omega(t_f))^{2/3}
\left[ \frac{\pi}{\dot{\omega}(t_f)} \right]^{1/2}
e^{i(2\pi ft_f - 2\Phi(t_f) - \pi/4)}
\ena
where $t_f$ is defined by equation
\be
\omega(t_f) = \pi f
\en
For restricted waveform it is sufficient to use only leading
order term in amplitude:
\be
(\dot{\omega}(t_f))^{-1/2} = M\sqrt{\frac{5}{96\eta}} 
(\pi Mf)^{-11/6}
\en
The phase (up to 2PN order) is given in \cite{DIS}:

\bea
2\pi ft_f - 2\Phi(t_f) &=& 2\pi ft_c - \phi_c + 
\frac3{128\eta}(\pi Mf)^{-5/3}\left\{ 
1 + \frac5{9}\left( \frac{743}{84} + 11\eta\right) - 16\pi
(\pi Mf) + \right.\nonumber\\
& & \left. \left( \frac{15293365}{1016064} + \frac{27145}{1008}\eta
+ \frac{3085}{144}\eta^2\right) (\pi Mf)^{4/3}
\right\} 
\ena
and inverting (\ref{fr}) we obtain
\bea
t_f  &=& t_c - \frac{5M}{256\eta}(\pi Mf)^{-8/3}\left\{
1 + \left( \frac{743}{252} + \frac{11}{3}\eta \right)(\pi Mf)^{2/3}
- \frac{32}{5}\pi (\pi Mf) + \right. \nonumber \\
& & \left. \left( \frac{8579163}{508032} + 
\frac{29555}{1008}\eta + \frac{203}{8}\eta^2\right) (\pi Mf)^{4/3}
\right\}
\ena

{\bf Confirm 2PN term in $t_f$}.




\begin{thebibliography}{}

\bibitem{BCV1} Buonanno, Chen, Vallisneri, Phys.Rev. D67 (2003) 
024016

\bibitem{Kidder} Kidder, Phys.Rev. D52 (1995) 821

\bibitem{FC} Finn, Chernoff, Phys.Rev. D47 (1993) 2198

\bibitem{BCV2} Buonanno, Chen, Vallisneri, Phys.Rev D67 (2003) 104025

\bibitem{BCPV} Buonanno, Chen, Pan, Vallisneri Phys.Rev. D70 (2004)
104003 

\bibitem{Blanchet} Blanchet, Faye, Iyer, Joguet Phys.Rev. D65 (2002) 061501

\bibitem{DIS} Damour, Iyer, Sathyaprakash Phys.Rev. D63 (2001) 044023

\bibitem{Cutler} Cutler, Phys.Rev. D57 (1998) 7089

\bibitem{ACST} Apostolatos, Cutler, Sussman, Thorne, Phys.Rev D49 (1994) 6274

\bibitem{KTV} Krolak,  Tinto, Vallisneri, Phys.Rev D70, (2004) 022003

\end{thebibliography}



\end{document}
